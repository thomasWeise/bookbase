%%
%% Access to Source Codes
%%
%
%
\newglossaryentry{android}{%
text={\softwareStyle{Android}},%
name={Android},%
sort={Android},%
description={%
is a common operating system for mobile phones~\cite{SGMS2022APTBNRG}.%
}%
}%
\protected\gdef\android{\pgls{android}}%
%
%
\newglossaryentry{aptGet}{%
text={\softwareStyle{apt\nobreakdashes-get}},%
name={apt-get},%
sort={apt-get},%
description={%
is a command used to install Debian~\textil{deb} packages under \ubuntu\ \linux~\cite{H2022LML,SFLR2009LIAN,VV2022LF}. %
Using \bashil{apt-get} requires superuser privileges~(\gls{sudo}), i.e., one usually does \bashil{sudo apt-get install...}. %
Learn more at \url{https://salsa.debian.org/apt-team/apt}.%
}%
}%
\protected\gdef\aptGet{\gls{aptGet}}%
%
%
\newglossaryentry{bash}{%
text={\softwareStyle{Bash}},%
name={Bash},%
sort={Bash},%
description={%
is a the shell used under \ubuntu\ \linux, i.e., the program that \inQuotes{runs} in the \gls{terminal} and interprets your commands, allowing you to start and interact with other programs~\cite{NR2005LTBSUSPCB3,Z2017MB,BN2018BC}. %
Learn more at \url{https://www.gnu.org/software/bash}.%
}%
}%
\protected\gdef\bash{\pgls{bash}}%
%
%
\newglossaryentry{debugger}{%
text={debugger},%
name={debugger},%
sort={debugger},%
description={%
A debugger is a tool that lets you execute a program step-by-step while observing the current values of variables. %
This allows you to find errors in the code more easily~\cite{W2024PME,A2002D,R2007PPBPDTAM}.%
\optionalRefElse{ut:debugger}{ See also \cref{ut:debugger}.}{ %
Learn more about debugging in~\cite{programmingWithPython}.}%
}%
}%
\protected\gdef\debugger{\pgls{debugger}}%
%
%
\newglossaryentry{docker}{%
text={\softwareStyle{Docker}},%
name={Docker},%
sort={Docker},%
description={%
provides \pgls{OS}-level virtualization. %
A Docker container is something like a more lightweight variant of a virtual machine. %
Docker containers offer a \linux\ runtime environment into which software can be installed and run isolated from the rest of the system. %
Such containers can be created by scripts. %
They offer a simple, reproducible, and portable way to configure and ship software components. %
Learn more at \url{https://docker.com} or in~\cite{LM2023DUR}.%
}%
}%
\protected\gdef\docker{\pgls{docker}}%
%
%
\newglossaryentry{flask}{%
text={\softwareStyle{Flask}},%
name={Flask},%
sort={Flask},%
description={%
is a lightweight \python\ framework that allows developers to quickly and easily build web applications~\cite{T2024MFWAAD,A2024FSFAR,CCSB2022HANNBHRSFOFWWRDARBAS}. %
It is based on the \python\ WSGI standard~\cite{PEP3333}. %
Learn more at \url{https://flask.palletsprojects.com}.%
}%
}%
%
%
\newglossaryentry{inkscape}{%
text={\softwareStyle{Inkscape}},%
name={Inkscape},%
sort={Inkscape},%
description={%
is a free and open source vector graphics editor, which primarily works with the \pgls{SVG} format~\cite{R2003DMEWI1APGTYJFBTPLVI,K2021TBOI}. %
This is the tool that I would recommend for professional graphic design. %
Learn more at \url{https://inkscape.org}.%
}%
}%
\protected\gdef\inkscape{\pgls{inkscape}}%
%
%
\newglossaryentry{appleIOS}{%
text={\softwareStyle{iOS}},%
name={iOS},%
sort={iOS},%
description={%
is the operating system that powers Apple iPhones~\cite{S2025MAPAMAIE2,C2024TCOI1AI1}. %
Learn more at \url{https://www.apple.com/ios}.%
}%
}%
\protected\gdef\appleIOS{\pgls{appleIOS}}%
%
%
\newglossaryentry{ibmDB2}{%
text={\softwareStyle{DB2}},%
name={DB2},%
sort={DB2},%
description={%
developed by IBM is one of the older and popular relational \pglspl{dbms}~\cite{HS2013THAGOID,CWDS2007UDLVWE,BBBCCDMMP2016SPTAFOIDFI}.%
}%
}%
\protected\gdef\ibmDB{\pgls{ibmDB2}}%
%
%
\newglossaryentry{iPadOS}{%
text={\softwareStyle{iPadOS}},%
name={iPadOS},%
sort={iPadOS},%
description={%
is the operating system that powers Apple iPads~\cite{C2024TCOI1AI1}. %
Learn more at \url{https://www.apple.com/ipados}.%
}%
}%
\protected\gdef\iPadOS{\pgls{iPadOS}}%
%
%
\newglossaryentry{lampStack}{%
text={\softwareStyle{LAMP} Stack},%
name={LAMP Stack},%
sort={LAMP Stack},%
description={%
A system setup for web applications: \linux, Apache~(a web \pgls{server}), \mysql, and the server-side scripting language PHP~\cite{C2022HAFTLS,H2020ULU2E}.%
}%
}%
\protected\gdef\lampStack{\pgls{lampStack}}%
%
%
\newglossaryentry{linux}{%
text={\softwareStyle{Linux}},%
name={Linux},%
sort={Linux},%
description={%
is the leading open source operating system, i.e., a free alternative for \microsoftWindows~\cite{T1999TLE,B2022ELATCL,H2022LML,SFLR2009LIAN,VV2022LF}. %
We recommend using it for this course, for software development, and for research. %
Learn more at \url{https://www.linux.org}. %
Its variant \ubuntu\ is particularly easy to use and install.%
}%
}%
\protected\gdef\linux{\pgls{linux}}%
%
%
\newglossaryentry{libreoffice}{%
text={LibreOffice},%
name={LibreOffice},%
sort={LibreOffice},%
description={%
is on open source office suite~\cite{DF2024LTDF,GL2012LTSOOSSCBAFACSOL,S2022L7PFEUU} which is a good and free alternative to \microsoftOffice. %
It offers software such as \libreofficeWriter, \libreofficeCalc, and \libreofficeBase.%
\optionalRefElse{sec:installLibreOffice}{ %
Installation instructions for \libreoffice\ are given in \cref{sec:installLibreOffice}.}{ %
See~\cite{databases} for more information and installation instructions.}%
}%
}%
\protected\gdef\libreoffice{\pgls{libreoffice}}%
%
%
\newglossaryentry{libreofficeBase}{%
text={LibreOffice~\softwareStyle{Base}},%
name={LibreOffice Base},%
sort={LibreOffice Base},%
description={%
is a \pglspl{dbms} that can work on stand-alone files but also connect to other popular \pglspl{rdb}~\cite{FNFHWSKLSSGLFRSRPLJG2022BG7R1BOL7C,S2022L7PFEUU}.
It is part of \gls{libreoffice}~\cite{DF2024LTDF,GL2012LTSOOSSCBAFACSOL,S2022L7PFEUU} and has functionality that is comparable to \microsoftAccess~\cite{SSI2023MA2BTA,B2020HOMA2,UC2021AFD}.%
\optionalRef{ut:libreOfficeBase}{ See also \cref{ut:libreOfficeBase}.}%
}%
}%
\protected\gdef\libreofficeBase{\pgls{libreofficeBase}}%
%
%
\newglossaryentry{libreofficeCalc}{%
text={LibreOffice~\softwareStyle{Calc}},%
name={LibreOffice Calc},%
sort={LibreOffice Calc},%
description={%
is a spreadsheet software that allows you to arrange and perform calculations with data in a tabular grid. %
It is a free and open source spread sheet software~\cite{S2022L7PFEUU,DF2024LTDF}, i.e., an alternative to \microsoftExcel. %
It is part of \gls{libreoffice}~\cite{DF2024LTDF,GL2012LTSOOSSCBAFACSOL,S2022L7PFEUU}.%
\optionalRef{fig:libreOfficeCalc}{ An example of how a \libreofficeCalc\ table looks like is given in~\cref{fig:libreOfficeCalc}.}%
}%
}%
\protected\gdef\libreofficeCalc{\pgls{libreofficeCalc}}%
%
%
\newglossaryentry{libreofficeWriter}{%
text={LibreOffice~\softwareStyle{Writer}},%
name={LibreOffice Writer},%
sort={LibreOffice Writer},%
description={%
is a free and open source text writing program~\cite{ZM2021AAFPDFAFMWALWFF} and part of \libreoffice~\cite{DF2024LTDF,GL2012LTSOOSSCBAFACSOL,S2022L7PFEUU}. %
It is a good alternative to \microsoftWord.%
}%
}%
\protected\gdef\libreofficeWriter{\pgls{libreofficeWriter}}%
%
%
\newglossaryentry{macOS}{%
text={\softwareStyle{macOS}},%
name={macOS},%
sort={macOS},%
description={%
or Mac~OS is the operating system that powers Apple Mac(intosh) computers~\cite{S2025MAPAMAIE2}. %
Learn more at \url{https://www.apple.com/macos}.%
}%
}%
\protected\gdef\macOS{\pgls{macOS}}%
%
%
\newglossaryentry{mariadb}{%
text={\softwareStyle{MariaDB}},%
name={MariaDB},%
sort={MariaDB},%
description={%
An open source \gls{rdb} management system that has forked off from \mysql~\cite{R2014MM,B2019LTMEELFFSAA,D2015LMAM,AA2018QAWMV1ITSQ,AA2018QAWMV2IDQ,M:MSD}. %
See \url{https://mariadb.org} for more information.%
}%
}%
\protected\gdef\mariadb{\pgls{mariadb}}%
%
%
\newglossaryentry{matplotlib}{%
text={\softwareStyle{Matplotlib}},%
name={Matplotlib},%
sort={Matplotlib},%
description={%
is a \python\ package for plotting diagrams and charts~\cite{HDFDM2012MVWP,H2007MA2GE,P2021HOMLPAVWP,J2018NPSCADSAWNSAM}. %
Learn more at at \url{https://matplotlib.org}~\cite{HDFDM2012MVWP}.%
}%
}%
\protected\gdef\matplotlib{\pgls{matplotlib}}%
%
%
\newglossaryentry{microsoftAccess}{%
text={Microsoft~\softwareStyle{Access}},%
name={Microsoft Access},%
sort={Microsoft Access},%
description={%
is a \pglspl{dbms} that can work on \pglspl{db} stored in single, stand-alone files but also connect to other popular \pglspl{rdb}~\cite{SSI2023MA2BTA,B2020HOMA2,UC2021AFD,MM2014RDAMA}. %
It is part of \microsoftOffice. %
A free and open source alternative to this commercial software is \libreofficeBase.%
}%
}%
\protected\gdef\microsoftAccess{\pgls{microsoftAccess}}%
%
%
\newglossaryentry{microsoftExcel}{%
text={Microsoft~\softwareStyle{Excel}},%
name={Microsoft Excel},%
sort={Microsoft Excel},%
description={%
is a spreadsheet program that allows users to store, organize, manipulate, and calculate data in tabular structures~\cite{B2023DMWME,G2024ECRFMME,LF2022MOSBSO2AM3}. %
It is part of \microsoftOffice. %
A free alternative to this commercial software is \libreofficeCalc~\cite{S2022L7PFEUU,DF2024LTDF}.%
}%
}%
\protected\gdef\microsoftExcel{\pgls{microsoftExcel}}%
%
\newglossaryentry{microsoftSqlServer}{%
text={Microsoft \softwareStyle{SQL~Server}},%
name={Microsoft SQL Server},%
sort={Microsoft SQL Server},%
description={%
The \microsoftSqlServer\ is a successful commercial relational/\sql-based \pgls{dbms}~\cite{P2020MSS2ABG,A2024TSAFMSS2,W2018MSSDB}. %
Learn more at \url{https://www.microsoft.com/sql-server} and \url{https://learn.microsoft.com/en-us/sql}.%
}%
}%
\protected\gdef\microsoftSqlServer{\pgls{microsoftSqlServer}}%
%
%
\newglossaryentry{microsoftWindows}{%
text={Microsoft \softwareStyle{Windows}},%
name={Microsoft Windows},%
sort={Microsoft Windows},%
description={%
is a commercial proprietary operating system~\cite{B2023W1IO}. %
It is widely spread, but we recommend using a \linux\ variant such as \ubuntu\ for software development and for our course. %
Learn more at \url{https://www.microsoft.com/windows}.%
}%
}%
\protected\gdef\microsoftWindows{\pgls{microsoftWindows}}%
%
%
\newglossaryentry{microsoftOffice}{%
text={Microsoft Office},%
name={Microsoft Office},%
sort={Microsoft Office},%
description={%
is a commercial suite of office software, including \microsoftExcel, \microsoftWord, and \microsoftAccess~\cite{LF2022MOSBSO2AM3}. %
\libreoffice\ is a free and open source alternative.%
}%
}%
\protected\gdef\microsoftOffice{\pgls{microsoftOffice}}%
%
%
\newglossaryentry{microsoftWord}{%
text={Microsoft~\softwareStyle{Word}},%
name={Microsoft Word},%
sort={Microsoft Word},%
description={%
is one of the leading text writing programs~\cite{MS2024MW,DR2019STFAWAUMW,ZM2021AAFPDFAFMWALWFF} and part of \microsoftOffice. %
A free alternative to this commercial software is the \libreofficeWriter.%
}%
}%
\protected\gdef\microsoftWord{\pgls{microsoftWord}}%
%
%
\newglossaryentry{moptipy}{%
text={\softwareStyle{moptipy}},%
name={moptipy},%
sort={moptipy},%
description={%
is the \emph{Metaheuristic Optimization in \python} library~\cite{WW2023RSDEWASSAA}. %
Learn more at \url{https://thomasweise.github.io/moptipy}.%
}%
}%
\protected\gdef\moptipy{\pgls{moptipy}}%
%
%
\newacronym[sort={MSYS2},description={
Minimal SYStem~2~(MSYS2) is a collection of tools and libraries from the \linux\ world providing an environment for building, installing, and running native \microsoftWindows\ software~\cite{TO2024BCCAIGTMCCAMC:MS}. %
Learn more at \url{https://www.msys2.org}.}]{MSYS2}{\softwareStyle{MSYS2}}{Minimal {SYStem}~2}%
%
%
\newglossaryentry{mypy}{%
text={\softwareStyle{Mypy}},%
name={Mypy},%
sort={Mypy},%
description={%
\pythonIdx{Mypy}%
is a static type checking tool for \python~\cite{LLHSVRZSJYYMC2024MOSTFP} that makes use of \pglspl{typeHint}. %
Learn more at \url{https://github.com/python/mypy}\optionalRefElse{sec:variableTypesAndTypeHints}{ %
or in \cref{sec:variableTypesAndTypeHints}}{ %
and in~\cite{programmingWithPython}}.%
}%
}%
\protected\gdef\mypy{\pgls{mypy}}%
%
%
\newglossaryentry{mysql}{%
text={\softwareStyle{MySQL}},%
name={MySQL},%
sort={MySQL},%
description={%
An open source \gls{rdb} management system~\cite{WAM2002MRMDFTS,TA2024DDAMWPAM,BT2021HPM,RGS2021BTOTONAMDFPC,D2015LMAM}. %
\mysql\ is famous for its use in the \gls{lampStack}. %
See \url{https://www.mysql.com} for more information.%
}%
}%
\protected\gdef\mysql{\pgls{mysql}}%
%
%
\newglossaryentry{mysqlWorkbench}{%
text={\softwareStyle{MySQL} Workbench},%
name={MySQL Workbench},%
sort={MySQL Workbench},%
description={%
is a visual tool for \db\ designers that offers tools ranging from graphical modeling to performance analysis~\cite{M2013MWDM}. %
Learn more at \url{https://www.mysql.com/products/workbench}.%
}%
}%
\protected\gdef\mysqlWorkbench{\gls{mysqlWorkbench}}%
%
%
\newglossaryentry{numpy}{%
text={\softwareStyle{NumPy}},%
name={NumPy},%
sort={NumPy},%
description={%
is a fundamental package for scientific computing with \python, which offers efficient array datastructures~\cite{HMvdWGVCWTBSKPHvKBHFdRWPGMSRWAGO2020APWN,DBvR2024ITN,J2018NPSCADSAWNSAM}. %
Learn more at \url{https://numpy.org}~\cite{N2025N}.%
}%
}%
\protected\gdef\numpy{\pgls{numpy}}%
%
%
\newglossaryentry{oracleDB}{%
text={\softwareStyle{Oracle} Database},%
name={Oracle Database},%
sort={Oracle Database},%
description={%
The \oracleDB\ was the first commercial \sql-based \pgls{rdb}~\cite{C20245YOQ}. %
It is still a highly successful proprietary product with many features~\cite{BBDDSY2011ADOODM,KK2021EODATASFHPAP}. %
Learn more at \url{https://www.oracle.com/database}.%
}%
}%
\protected\gdef\oracleDB{\pgls{oracleDB}}%
%
%
\newglossaryentry{pandas}{%
text={\softwareStyle{Pandas}},%
name={Pandas},%
sort={Pandas},%
description={%
is a \python\ data analysis and manipulation library~\cite{B2012DPWP,L2024PW}. %
Learn more at \url{https://pandas.pydata.org}~\cite{PD2025P}.%
}%
}%
\protected\gdef\pandas{\pgls{pandas}}%
%
%
\newglossaryentry{pgmodeler}{%
text={\softwareStyle{PgModeler}},%
name={PgModeler},%
sort={PgModeler},%
description={%
the \postgresql\ \db\ modeler is a tool that allows for graphical modeling of logical schemas for \pglspl{db} using an \pgls{ERD}-like notation~\cite{AES2006PPDM}. %
Learn more at \url{https://pgmodeler.io}.%
}%
}%
\protected\gdef\pgmodeler{\gls{pgmodeler}}%
%
%
\newglossaryentry{pip}{%
text={\softwareStyle{pip}},%
name={pip},%
sort={pip},%
description={%
\pythonIdx{pip}%
is the standard tool to install \python\ software packages from the \pgls{pypi} repository~\cite{PSF:P3D:IPM,PD2024PD}. %
To install a package \bashil{thepackage} hosted on \pgls{pypi}, type \bashil{pip install thepackage} into the \gls{terminal}. %
Learn more at \url{https://packaging.python.org/installing}.%
}%
}%
\protected\gdef\pip{\gls{pip}}%
%
%
\newglossaryentry{postgresql}{%
text={\softwareStyle{PostgreSQL}},%
name={PostgreSQL},%
sort={PostgreSQL},%
description={%
An open source object-relational \pgls{dbms}~\cite{TA2024DDAMWPAM,FP2023LP,OH2017PUAR,B2024PELUYDW}. %
See \url{https://postgresql.org}\optionalRef{ut:postgresql}{ and \cref{ut:postgresql}} for more information.%
}%
}%
\protected\gdef\postgresql{\gls{postgresql}}%
%
%
\newglossaryentry{psql}{%
text={\softwareStyle{psql}},%
name={psql},%
sort={psql},%
description={%
is the \gls{client} program used to access the \postgresql\ \gls{dbms} server.%
\optionalRef{ut:psql}{ See also \cref{ut:psql}.}%
}%
}%
\protected\gdef\psql{\pgls{psql}}%
%
%
\newglossaryentry{psycopg}{%
text={\softwareStyle{psycopg}},%
name={psycopg},%
sort={psycopg},%
description={%
or, more exactly, \pythonil{psycopg 3}, is the most popular \postgresql\ adapter for \python, implementing the \python~\acrshort{db}~\acrshort{API}~2.0 specification~\cite{PEP249}. %
Learn more at \url{https://www.psycopg.org}~\cite{VDGE2010P}\optionalRef{ut:psycopg}{ and \cref{ut:psycopg}}.%
}%
}%
\protected\gdef\psycopg{\pgls{psycopg}}%
%
%
\newglossaryentry{pycharm}{%
text={\softwareStyle{PyCharm}},%
name={PyCharm},%
sort={PyCharm},%
description={%
is the convenient \python\ \gls{ide} that we recommend for this course~\cite{VHN2023HOADWP,Y2022PPADT,W2024PME}. %
It comes in a free community edition, so it can be downloaded and used at no cost. %
Learn more at \url{https://www.jetbrains.com/pycharm}.%
}%
}%
\protected\gdef\pycharm{\pgls{pycharm}}%
%
%
\newglossaryentry{pylint}{%
text={\softwareStyle{Pylint}},%
name={Pylint},%
sort={Pylint},%
description={%
is a \gls{linter} for \python\ that checks for errors, enforces coding standards, and that can make suggestions for improvements~\cite{PC2024PL}. %
Learn more at \url{https://www.pylint.org}\optionalRefElse{ut:pylint}{ and \cref{ut:pylint}{ and in~\cite{programmingWithPython}}}.%
}%
}%
\protected\gdef\pylint{\pgls{pylint}}%
%
%
\newglossaryentry{pytest}{%
text={\softwareStyle{pytest}},%
name={pytest},%
sort={pytest},%
description={%
is a framework for writing and executing \pglspl{unitTest} in \python~\cite{KPDT2024PD,O2022PTWP,DG2020TIP,P2021PUTAAOAEUTIP,W2024PME}. %
Learn more at \url{https://pytest.org}\optionalRefElse{sec:unitTesting}{ and in \cref{sec:unitTesting}{ and in~\cite{programmingWithPython}}}.%
}%
}%
\protected\gdef\pytest{\pgls{pytest}}%
%
%
\newglossaryentry{python}{%
text={\softwareStyle{Python}},%
name={Python},%
sort={Python},%
description={The \softwareStyle{Python} programming language~\cite{H2023ABGTP3P,LH2015DSAAWP,programmingWithPython,L2025LP}, i.e., what you will learn about in our book~\cite{programmingWithPython}. %
Learn more at \url{https://python.org}.}
}%
\protected\gdef\python{\pgls{python}}%
%
\xdef\pythonVersion{3.12}%
\protected\gdef\pythonWithVersion{\softwareStyle{\python~\pythonVersion}}%
%
%
\newglossaryentry{pypi}{%
text={\softwareStyle{PyPI}},%
name={PyPI},%
sort={PyPI},%
description={%
The \python\ Package Index~(PyPI) is an online repository that provides the software packages that you can install with~\pip~\cite{PSF:TPPIP,BB2019AEAOTPPIP,VVH2018ELDOSAIOSPACSOTPE}. %
Learn more at \url{https://pypi.org}.%
}%
}%
\protected\gdef\pypi{\pgls{pypi}}%
%
%
\newglossaryentry{pytorch}{%
text={\softwareStyle{PyTorch}},%
name={PyTorch},%
sort={PyTorch},%
description={is a \python\ library for deep learning and \gls{AI}~\cite{PGMLBCKLGADKYDRTCSFBC2019PAISHPDLL,RLM2022MLWPAS}. %
Learn more at \url{https://pytorch.org}.}
}%
\protected\gdef\pytorch{\pgls{pytorch}}%
%
%
\newglossaryentry{ruff}{%
text={\softwareStyle{Ruff}},%
name={Ruff},%
sort={Ruff},%
description={%
\pythonIdx{ruff}%
is a \gls{linter} and code formatting tool for \python~\cite{M2022RAEFPLACFWIR,PSF:TPPIP:R}. %
Learn more at \url{https://docs.astral.sh/ruff}\optionalRefElse{ut:ruff}{ or in \cref{ut:ruff}}{ or in~\cite{programmingWithPython}}.%
}%
}%
\protected\gdef\ruff{\pgls{ruff}}%
%
%
\newglossaryentry{scikitlearn}{%
text={\softwareStyle{Scikit-learn}},%
name={Scikit-learn},%
sort={Scikit-learn},%
description={is a \python\ library offering various machine learning tools~\cite{PVGMTGBPWDVPCBPD2011SMLIP,RLM2022MLWPAS}. %
Learn more at \url{https://scikit-learn.org}.}
}%
\protected\gdef\scikitlearn{\pgls{scikitlearn}}%
%
%
\newglossaryentry{scipy}{%
text={\softwareStyle{SciPy}},%
name={SciPy},%
sort={SciPy},%
description={is a \python\ library for scientific computing~\cite{VGOHRCBPWBvdWBWMMNJKLCPFMVLPCHQHARPvMS2020SFAFSCIP,J2018NPSCADSAWNSAM}. %
Learn more at \url{https://scipy.org}.}
}%
\protected\gdef\scipy{\pgls{scipy}}%
%
%
\newglossaryentry{simpy}{%
text={\softwareStyle{SimPy}},%
name={SimPy},%
sort={SimPy},%
description={is a \python\ library for discrete event simulation~\cite{Z2024DESIEWS}. %
Learn more at \url{https://simpy.readthedocs.io}.}
}%
\protected\gdef\simpy{\pgls{simpy}}%
%
%
\newglossaryentry{sqlite}{%
text={\softwareStyle{SQLite}},%
name={SQLite},%
sort={SQLite},%
description={is an relational \gls{dbms} which runs as in-process library that works directly on files as opposed to the \gls{clientServerArchitecture} used by other common \pglspl{dbms}. %
It is the most wide-spread \gls{SQL}-based \gls{db} in use today, installed in nearly every smartphone, computer, web browser, television, and automobile~\cite{WB2019RHSOOS,GPBHKP2022SPPAF,C20245YOQ,HWACIS:HO2023WKUOS}. %
Learn more at \url{https://sqlite.org}~\cite{HWACIS}.}
}%
\protected\gdef\sqlite{\pgls{sqlite}}%
%
%
\newglossaryentry{sphinx}{%
text={\softwareStyle{Sphinx}},%
name={Sphinx},%
sort={Sphinx},%
description={\sphinx\ is a tool for generating software documentation~\cite{TTSFABFSKSD2024SCIABDWE}. %
It supports \python\ can use both \pglspl{docstring} and \pglspl{typeHint} to generate beautiful documents. %
Learn more at \url{https://www.sphinx-doc.org}.}
}%
\protected\gdef\sphinx{\pgls{sphinx}}%
%
%
\newglossaryentry{tensorflow}{%
text={\softwareStyle{TensorFlow}},%
name={TensorFlow},%
sort={TensorFlow},%
description={is a \python\ library for implementing machine learning, especially suitable for training of neural networks~\cite{ABCCDDDGIIKLMMMSTVWWYZ2016TASFLSML,L2023TDDBTADMLMWT}. %
Learn more at \url{https://www.tensorflow.org}.}
}%
\protected\gdef\tensorflow{\pgls{tensorflow}}%
%
%
\newglossaryentry{ubuntu}{%
text={\softwareStyle{Ubuntu}},%
name={Ubuntu},%
sort={Ubuntu},%
description={%
is a variant of the open source operating system \linux~\cite{CN2020ULB,H2020ULU2E}. %
We recommend that you use this operating system to follow this class, for software development, and for research. %
Learn more at \url{https://ubuntu.com}. %
If you are in China, you can download it from \expandafter\url{\ubuntuDownloadUrl}.%
}%
}%
\protected\gdef\ubuntu{\pgls{ubuntu}}%
\xdef\ubuntuDownloadUrl{https://mirrors.ustc.edu.cn/ubuntu-releases}%
%
%
\newglossaryentry{venv}{%
text={\softwareStyle{venv}},%
name={venv},%
sort={venv},%
description={%
\pythonIdx{venv}%
is a \python\ module and tool for creating \pglspl{virtualEnvironment}~\cite{PSF:P3D:TPSL:VCOVE}. %
Learn more at \url{https://docs.python.org/3/library/venv.html}~\cite{PSF:P3D:TPSL:VCOVE}.%
}%
}%
\protected\gdef\venv{\pgls{venv}}%
%
%
\newglossaryentry{yEd}{%
text={\softwareStyle{yEd}},%
name={yEd},%
sort={yEd},%
description={%
is a graph editor for high-quality graph-based diagrams~\cite{SG2015MDAWY,Y2011YGEM}, suitable to draw, e.g., technology-independent \pglspl{ERD}, control flow charts, or \pgls{UML} class diagrams. %
An online version of the editor is available at \url{https://www.yworks.com/yed-live}. %
Learn more at \url{https://www.yworks.com/products/yed}.%
\optionalRef{ut:yEd}{ See also \cref{ut:yEd}.}%
}%
}%
\protected\gdef\yEd{\pgls{yEd}}%
%

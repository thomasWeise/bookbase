%%
%% Commands for Definitions.
%%
\let\th@plain\relax%%%
\RequirePackage{ntheorem}%
\RequirePackage[framemethod=TikZ]{mdframed}%
%
\theorembodyfont{\normalfont}%
\newcommand{\definitionautorefname}{Definition}%%
%
\ifIsSlides\expandafter\@firstoftwo\else\expandafter\@secondoftwo\fi{%
\theoremstyle{nonumberplain}%
}{}%
%
%% Definitions
%% Definitions provide boxes in which we can define stuff.
\mdfdefinestyle{definitionStyle}{%
linewidth=2pt,%
linecolor=definition-frame,%
roundcorner=4pt,%
backgroundcolor=definition-background,%
leftmargin=5pt,%
rightmargin=5pt,%
font={},%
nobreak=true%
}%
%
%%% Define the definition environment based on the slides/book situation.
\ifIsSlides\expandafter\@firstoftwo\else\expandafter\@secondoftwo\fi{%
\mdtheorem[style=definitionStyle]{definition}{\definitionautorefname}%
}{%
\mdtheorem[style=definitionStyle]{definition}{\definitionautorefname}[chapter]%
}%
%
%% Setting up the proper title
\xpatchcmd{\definition}{\refstepcounter}{\NR@gettitle{#1}\refstepcounter}{}{}%
%
%
%%% Best Practices
%% The best practices suite offers the command \bestPractice that prints
%% information  about a best practice in a framed box.
%% The list of all best practices can be printed via \printBestPractices.
\newcommand{\@bestPracticeautorefname}{Best Practice}%
\mdfdefinestyle{@bestPracticeStyle}{%
linewidth=2pt,%
linecolor=bestpractice-frame,%
roundcorner=4pt,%
backgroundcolor=bestpractice-background,%
leftmargin=5pt,%
rightmargin=5pt,%
font={},%
nobreak=true%
}%
\mdtheorem[style=@bestPracticeStyle]{@bestPractice}{\@bestPracticeautorefname}%
%% Setting up the proper title
\xpatchcmd{\@bestPractice}{\refstepcounter}{\NR@gettitle{#1}\refstepcounter}{}{}%
%
%
%%% Useful Tools
%% The useful tools command suite offers the command \usefulTool that
%% prints information about a useful tool in a framed box.
%% The list of all useful tools can be printed via \printUsefulTools.
\newcommand{\@usefulToolautorefname}{Useful Tool}%
\mdfdefinestyle{@usefulToolStyle}{%
linewidth=2pt,%
linecolor=usefultool-frame,%
roundcorner=4pt,%
backgroundcolor=usefultool-background,%
leftmargin=5pt,%
rightmargin=5pt,%
font={},%
nobreak=true%
}%
\mdtheorem[style=@usefulToolStyle]{@usefulTool}{\@usefulToolautorefname}%
%% Setting up the proper title
\xpatchcmd{\@usefulTool}{\refstepcounter}{\NR@gettitle{#1}\refstepcounter}{}{}%
%
%
%%% A Novice Hint is a box intended to give some useful information for novies
%%% or first-time readers.
\protected\gdef\noviceHint#1{%
\begin{center}%
\fcolorbox{novice-hint-frame}{novice-hint-background}{%
\parbox{0.9\linewidth}{%
\textbf{First Time Readers and Novices:}~#1
}}%
\end{center}%
}%
%
\ifIsBook\expandafter\@firstoftwo\else\expandafter\@secondoftwo\fi{%
%%
%% Commands for Definitions.
%%
\let\th@plain\relax%%%
\RequirePackage{ntheorem}%
\RequirePackage[framemethod=TikZ]{mdframed}%
%
\theorembodyfont{\normalfont}%
\newcommand{\definitionautorefname}{Definition}%%
%
\ifIsSlides\expandafter\@firstoftwo\else\expandafter\@secondoftwo\fi{%
\theoremstyle{nonumberplain}%
}{}%
%
%% Definitions
%% Definitions provide boxes in which we can define stuff.
\mdfdefinestyle{definitionStyle}{%
linewidth=2pt,%
linecolor=definition-frame,%
roundcorner=4pt,%
backgroundcolor=definition-background,%
leftmargin=5pt,%
rightmargin=5pt,%
font={},%
nobreak=true%
}%
%
%%% Define the definition environment based on the slides/book situation.
\ifIsSlides\expandafter\@firstoftwo\else\expandafter\@secondoftwo\fi{%
\mdtheorem[style=definitionStyle]{definition}{\definitionautorefname}%
}{%
\mdtheorem[style=definitionStyle]{definition}{\definitionautorefname}[chapter]%
}%
%
%% Setting up the proper title
\xpatchcmd{\definition}{\refstepcounter}{\NR@gettitle{#1}\refstepcounter}{}{}%
%
%
%%% Best Practices
%% The best practices suite offers the command \bestPractice that prints
%% information  about a best practice in a framed box.
%% The list of all best practices can be printed via \printBestPractices.
\newcommand{\@bestPracticeautorefname}{Best Practice}%
\mdfdefinestyle{@bestPracticeStyle}{%
linewidth=2pt,%
linecolor=bestpractice-frame,%
roundcorner=4pt,%
backgroundcolor=bestpractice-background,%
leftmargin=5pt,%
rightmargin=5pt,%
font={},%
nobreak=true%
}%
\mdtheorem[style=@bestPracticeStyle]{@bestPractice}{\@bestPracticeautorefname}%
%% Setting up the proper title
\xpatchcmd{\@bestPractice}{\refstepcounter}{\NR@gettitle{#1}\refstepcounter}{}{}%
%
%
%%% Useful Tools
%% The useful tools command suite offers the command \usefulTool that
%% prints information about a useful tool in a framed box.
%% The list of all useful tools can be printed via \printUsefulTools.
\newcommand{\@usefulToolautorefname}{Useful Tool}%
\mdfdefinestyle{@usefulToolStyle}{%
linewidth=2pt,%
linecolor=usefultool-frame,%
roundcorner=4pt,%
backgroundcolor=usefultool-background,%
leftmargin=5pt,%
rightmargin=5pt,%
font={},%
nobreak=true%
}%
\mdtheorem[style=@usefulToolStyle]{@usefulTool}{\@usefulToolautorefname}%
%% Setting up the proper title
\xpatchcmd{\@usefulTool}{\refstepcounter}{\NR@gettitle{#1}\refstepcounter}{}{}%
%
%
%%% A Novice Hint is a box intended to give some useful information for novies
%%% or first-time readers.
\protected\gdef\noviceHint#1{%
\begin{center}%
\fcolorbox{novice-hint-frame}{novice-hint-background}{%
\parbox{0.9\linewidth}{%
\textbf{First Time Readers and Novices:}~#1
}}%
\end{center}%
}%
%
\ifIsBook\expandafter\@firstoftwo\else\expandafter\@secondoftwo\fi{%
%%
%% Commands for Definitions.
%%
\let\th@plain\relax%%%
\RequirePackage{ntheorem}%
\RequirePackage[framemethod=TikZ]{mdframed}%
%
\theorembodyfont{\normalfont}%
\newcommand{\definitionautorefname}{Definition}%%
%
\ifIsSlides\expandafter\@firstoftwo\else\expandafter\@secondoftwo\fi{%
\theoremstyle{nonumberplain}%
}{}%
%
%% Definitions
%% Definitions provide boxes in which we can define stuff.
\mdfdefinestyle{definitionStyle}{%
linewidth=2pt,%
linecolor=definition-frame,%
roundcorner=4pt,%
backgroundcolor=definition-background,%
leftmargin=5pt,%
rightmargin=5pt,%
font={},%
nobreak=true%
}%
%
%%% Define the definition environment based on the slides/book situation.
\ifIsSlides\expandafter\@firstoftwo\else\expandafter\@secondoftwo\fi{%
\mdtheorem[style=definitionStyle]{definition}{\definitionautorefname}%
}{%
\mdtheorem[style=definitionStyle]{definition}{\definitionautorefname}[chapter]%
}%
%
%% Setting up the proper title
\xpatchcmd{\definition}{\refstepcounter}{\NR@gettitle{#1}\refstepcounter}{}{}%
%
%
%%% Best Practices
%% The best practices suite offers the command \bestPractice that prints
%% information  about a best practice in a framed box.
%% The list of all best practices can be printed via \printBestPractices.
\newcommand{\@bestPracticeautorefname}{Best Practice}%
\mdfdefinestyle{@bestPracticeStyle}{%
linewidth=2pt,%
linecolor=bestpractice-frame,%
roundcorner=4pt,%
backgroundcolor=bestpractice-background,%
leftmargin=5pt,%
rightmargin=5pt,%
font={},%
nobreak=true%
}%
\mdtheorem[style=@bestPracticeStyle]{@bestPractice}{\@bestPracticeautorefname}%
%% Setting up the proper title
\xpatchcmd{\@bestPractice}{\refstepcounter}{\NR@gettitle{#1}\refstepcounter}{}{}%
%
%
%%% Useful Tools
%% The useful tools command suite offers the command \usefulTool that
%% prints information about a useful tool in a framed box.
%% The list of all useful tools can be printed via \printUsefulTools.
\newcommand{\@usefulToolautorefname}{Useful Tool}%
\mdfdefinestyle{@usefulToolStyle}{%
linewidth=2pt,%
linecolor=usefultool-frame,%
roundcorner=4pt,%
backgroundcolor=usefultool-background,%
leftmargin=5pt,%
rightmargin=5pt,%
font={},%
nobreak=true%
}%
\mdtheorem[style=@usefulToolStyle]{@usefulTool}{\@usefulToolautorefname}%
%% Setting up the proper title
\xpatchcmd{\@usefulTool}{\refstepcounter}{\NR@gettitle{#1}\refstepcounter}{}{}%
%
%
%%% A Novice Hint is a box intended to give some useful information for novies
%%% or first-time readers.
\protected\gdef\noviceHint#1{%
\begin{center}%
\fcolorbox{novice-hint-frame}{novice-hint-background}{%
\parbox{0.9\linewidth}{%
\textbf{First Time Readers and Novices:}~#1
}}%
\end{center}%
}%
%
\ifIsBook\expandafter\@firstoftwo\else\expandafter\@secondoftwo\fi{%
%%
%% Commands for Definitions.
%%
\let\th@plain\relax%%%
\RequirePackage{ntheorem}%
\RequirePackage[framemethod=TikZ]{mdframed}%
%
\theorembodyfont{\normalfont}%
\newcommand{\definitionautorefname}{Definition}%%
%
\ifIsSlides\expandafter\@firstoftwo\else\expandafter\@secondoftwo\fi{%
\theoremstyle{nonumberplain}%
}{}%
%
%% Definitions
%% Definitions provide boxes in which we can define stuff.
\mdfdefinestyle{definitionStyle}{%
linewidth=2pt,%
linecolor=definition-frame,%
roundcorner=4pt,%
backgroundcolor=definition-background,%
leftmargin=5pt,%
rightmargin=5pt,%
font={},%
nobreak=true%
}%
%
%%% Define the definition environment based on the slides/book situation.
\ifIsSlides\expandafter\@firstoftwo\else\expandafter\@secondoftwo\fi{%
\mdtheorem[style=definitionStyle]{definition}{\definitionautorefname}%
}{%
\mdtheorem[style=definitionStyle]{definition}{\definitionautorefname}[chapter]%
}%
%
%% Setting up the proper title
\xpatchcmd{\definition}{\refstepcounter}{\NR@gettitle{#1}\refstepcounter}{}{}%
%
%
%%% Best Practices
%% The best practices suite offers the command \bestPractice that prints
%% information  about a best practice in a framed box.
%% The list of all best practices can be printed via \printBestPractices.
\newcommand{\@bestPracticeautorefname}{Best Practice}%
\mdfdefinestyle{@bestPracticeStyle}{%
linewidth=2pt,%
linecolor=bestpractice-frame,%
roundcorner=4pt,%
backgroundcolor=bestpractice-background,%
leftmargin=5pt,%
rightmargin=5pt,%
font={},%
nobreak=true%
}%
\mdtheorem[style=@bestPracticeStyle]{@bestPractice}{\@bestPracticeautorefname}%
%% Setting up the proper title
\xpatchcmd{\@bestPractice}{\refstepcounter}{\NR@gettitle{#1}\refstepcounter}{}{}%
%
%
%%% Useful Tools
%% The useful tools command suite offers the command \usefulTool that
%% prints information about a useful tool in a framed box.
%% The list of all useful tools can be printed via \printUsefulTools.
\newcommand{\@usefulToolautorefname}{Useful Tool}%
\mdfdefinestyle{@usefulToolStyle}{%
linewidth=2pt,%
linecolor=usefultool-frame,%
roundcorner=4pt,%
backgroundcolor=usefultool-background,%
leftmargin=5pt,%
rightmargin=5pt,%
font={},%
nobreak=true%
}%
\mdtheorem[style=@usefulToolStyle]{@usefulTool}{\@usefulToolautorefname}%
%% Setting up the proper title
\xpatchcmd{\@usefulTool}{\refstepcounter}{\NR@gettitle{#1}\refstepcounter}{}{}%
%
%
%%% A Novice Hint is a box intended to give some useful information for novies
%%% or first-time readers.
\protected\gdef\noviceHint#1{%
\begin{center}%
\fcolorbox{novice-hint-frame}{novice-hint-background}{%
\parbox{0.9\linewidth}{%
\textbf{First Time Readers and Novices:}~#1
}}%
\end{center}%
}%
%
\ifIsBook\expandafter\@firstoftwo\else\expandafter\@secondoftwo\fi{%
\input{\bookbaseDir/styles/books/definitions.tex}%
}{}%
\ifIsSlides\expandafter\@firstoftwo\else\expandafter\@secondoftwo\fi{%
\input{\bookbaseDir/styles/slides/definitions.tex}%
}{}%
%
%
}{}%
\ifIsSlides\expandafter\@firstoftwo\else\expandafter\@secondoftwo\fi{%
%%
%% Commands for Definitions.
%%
\let\th@plain\relax%%%
\RequirePackage{ntheorem}%
\RequirePackage[framemethod=TikZ]{mdframed}%
%
\theorembodyfont{\normalfont}%
\newcommand{\definitionautorefname}{Definition}%%
%
\ifIsSlides\expandafter\@firstoftwo\else\expandafter\@secondoftwo\fi{%
\theoremstyle{nonumberplain}%
}{}%
%
%% Definitions
%% Definitions provide boxes in which we can define stuff.
\mdfdefinestyle{definitionStyle}{%
linewidth=2pt,%
linecolor=definition-frame,%
roundcorner=4pt,%
backgroundcolor=definition-background,%
leftmargin=5pt,%
rightmargin=5pt,%
font={},%
nobreak=true%
}%
%
%%% Define the definition environment based on the slides/book situation.
\ifIsSlides\expandafter\@firstoftwo\else\expandafter\@secondoftwo\fi{%
\mdtheorem[style=definitionStyle]{definition}{\definitionautorefname}%
}{%
\mdtheorem[style=definitionStyle]{definition}{\definitionautorefname}[chapter]%
}%
%
%% Setting up the proper title
\xpatchcmd{\definition}{\refstepcounter}{\NR@gettitle{#1}\refstepcounter}{}{}%
%
%
%%% Best Practices
%% The best practices suite offers the command \bestPractice that prints
%% information  about a best practice in a framed box.
%% The list of all best practices can be printed via \printBestPractices.
\newcommand{\@bestPracticeautorefname}{Best Practice}%
\mdfdefinestyle{@bestPracticeStyle}{%
linewidth=2pt,%
linecolor=bestpractice-frame,%
roundcorner=4pt,%
backgroundcolor=bestpractice-background,%
leftmargin=5pt,%
rightmargin=5pt,%
font={},%
nobreak=true%
}%
\mdtheorem[style=@bestPracticeStyle]{@bestPractice}{\@bestPracticeautorefname}%
%% Setting up the proper title
\xpatchcmd{\@bestPractice}{\refstepcounter}{\NR@gettitle{#1}\refstepcounter}{}{}%
%
%
%%% Useful Tools
%% The useful tools command suite offers the command \usefulTool that
%% prints information about a useful tool in a framed box.
%% The list of all useful tools can be printed via \printUsefulTools.
\newcommand{\@usefulToolautorefname}{Useful Tool}%
\mdfdefinestyle{@usefulToolStyle}{%
linewidth=2pt,%
linecolor=usefultool-frame,%
roundcorner=4pt,%
backgroundcolor=usefultool-background,%
leftmargin=5pt,%
rightmargin=5pt,%
font={},%
nobreak=true%
}%
\mdtheorem[style=@usefulToolStyle]{@usefulTool}{\@usefulToolautorefname}%
%% Setting up the proper title
\xpatchcmd{\@usefulTool}{\refstepcounter}{\NR@gettitle{#1}\refstepcounter}{}{}%
%
%
%%% A Novice Hint is a box intended to give some useful information for novies
%%% or first-time readers.
\protected\gdef\noviceHint#1{%
\begin{center}%
\fcolorbox{novice-hint-frame}{novice-hint-background}{%
\parbox{0.9\linewidth}{%
\textbf{First Time Readers and Novices:}~#1
}}%
\end{center}%
}%
%
\ifIsBook\expandafter\@firstoftwo\else\expandafter\@secondoftwo\fi{%
\input{\bookbaseDir/styles/books/definitions.tex}%
}{}%
\ifIsSlides\expandafter\@firstoftwo\else\expandafter\@secondoftwo\fi{%
\input{\bookbaseDir/styles/slides/definitions.tex}%
}{}%
%
%
}{}%
%
%
}{}%
\ifIsSlides\expandafter\@firstoftwo\else\expandafter\@secondoftwo\fi{%
%%
%% Commands for Definitions.
%%
\let\th@plain\relax%%%
\RequirePackage{ntheorem}%
\RequirePackage[framemethod=TikZ]{mdframed}%
%
\theorembodyfont{\normalfont}%
\newcommand{\definitionautorefname}{Definition}%%
%
\ifIsSlides\expandafter\@firstoftwo\else\expandafter\@secondoftwo\fi{%
\theoremstyle{nonumberplain}%
}{}%
%
%% Definitions
%% Definitions provide boxes in which we can define stuff.
\mdfdefinestyle{definitionStyle}{%
linewidth=2pt,%
linecolor=definition-frame,%
roundcorner=4pt,%
backgroundcolor=definition-background,%
leftmargin=5pt,%
rightmargin=5pt,%
font={},%
nobreak=true%
}%
%
%%% Define the definition environment based on the slides/book situation.
\ifIsSlides\expandafter\@firstoftwo\else\expandafter\@secondoftwo\fi{%
\mdtheorem[style=definitionStyle]{definition}{\definitionautorefname}%
}{%
\mdtheorem[style=definitionStyle]{definition}{\definitionautorefname}[chapter]%
}%
%
%% Setting up the proper title
\xpatchcmd{\definition}{\refstepcounter}{\NR@gettitle{#1}\refstepcounter}{}{}%
%
%
%%% Best Practices
%% The best practices suite offers the command \bestPractice that prints
%% information  about a best practice in a framed box.
%% The list of all best practices can be printed via \printBestPractices.
\newcommand{\@bestPracticeautorefname}{Best Practice}%
\mdfdefinestyle{@bestPracticeStyle}{%
linewidth=2pt,%
linecolor=bestpractice-frame,%
roundcorner=4pt,%
backgroundcolor=bestpractice-background,%
leftmargin=5pt,%
rightmargin=5pt,%
font={},%
nobreak=true%
}%
\mdtheorem[style=@bestPracticeStyle]{@bestPractice}{\@bestPracticeautorefname}%
%% Setting up the proper title
\xpatchcmd{\@bestPractice}{\refstepcounter}{\NR@gettitle{#1}\refstepcounter}{}{}%
%
%
%%% Useful Tools
%% The useful tools command suite offers the command \usefulTool that
%% prints information about a useful tool in a framed box.
%% The list of all useful tools can be printed via \printUsefulTools.
\newcommand{\@usefulToolautorefname}{Useful Tool}%
\mdfdefinestyle{@usefulToolStyle}{%
linewidth=2pt,%
linecolor=usefultool-frame,%
roundcorner=4pt,%
backgroundcolor=usefultool-background,%
leftmargin=5pt,%
rightmargin=5pt,%
font={},%
nobreak=true%
}%
\mdtheorem[style=@usefulToolStyle]{@usefulTool}{\@usefulToolautorefname}%
%% Setting up the proper title
\xpatchcmd{\@usefulTool}{\refstepcounter}{\NR@gettitle{#1}\refstepcounter}{}{}%
%
%
%%% A Novice Hint is a box intended to give some useful information for novies
%%% or first-time readers.
\protected\gdef\noviceHint#1{%
\begin{center}%
\fcolorbox{novice-hint-frame}{novice-hint-background}{%
\parbox{0.9\linewidth}{%
\textbf{First Time Readers and Novices:}~#1
}}%
\end{center}%
}%
%
\ifIsBook\expandafter\@firstoftwo\else\expandafter\@secondoftwo\fi{%
%%
%% Commands for Definitions.
%%
\let\th@plain\relax%%%
\RequirePackage{ntheorem}%
\RequirePackage[framemethod=TikZ]{mdframed}%
%
\theorembodyfont{\normalfont}%
\newcommand{\definitionautorefname}{Definition}%%
%
\ifIsSlides\expandafter\@firstoftwo\else\expandafter\@secondoftwo\fi{%
\theoremstyle{nonumberplain}%
}{}%
%
%% Definitions
%% Definitions provide boxes in which we can define stuff.
\mdfdefinestyle{definitionStyle}{%
linewidth=2pt,%
linecolor=definition-frame,%
roundcorner=4pt,%
backgroundcolor=definition-background,%
leftmargin=5pt,%
rightmargin=5pt,%
font={},%
nobreak=true%
}%
%
%%% Define the definition environment based on the slides/book situation.
\ifIsSlides\expandafter\@firstoftwo\else\expandafter\@secondoftwo\fi{%
\mdtheorem[style=definitionStyle]{definition}{\definitionautorefname}%
}{%
\mdtheorem[style=definitionStyle]{definition}{\definitionautorefname}[chapter]%
}%
%
%% Setting up the proper title
\xpatchcmd{\definition}{\refstepcounter}{\NR@gettitle{#1}\refstepcounter}{}{}%
%
%
%%% Best Practices
%% The best practices suite offers the command \bestPractice that prints
%% information  about a best practice in a framed box.
%% The list of all best practices can be printed via \printBestPractices.
\newcommand{\@bestPracticeautorefname}{Best Practice}%
\mdfdefinestyle{@bestPracticeStyle}{%
linewidth=2pt,%
linecolor=bestpractice-frame,%
roundcorner=4pt,%
backgroundcolor=bestpractice-background,%
leftmargin=5pt,%
rightmargin=5pt,%
font={},%
nobreak=true%
}%
\mdtheorem[style=@bestPracticeStyle]{@bestPractice}{\@bestPracticeautorefname}%
%% Setting up the proper title
\xpatchcmd{\@bestPractice}{\refstepcounter}{\NR@gettitle{#1}\refstepcounter}{}{}%
%
%
%%% Useful Tools
%% The useful tools command suite offers the command \usefulTool that
%% prints information about a useful tool in a framed box.
%% The list of all useful tools can be printed via \printUsefulTools.
\newcommand{\@usefulToolautorefname}{Useful Tool}%
\mdfdefinestyle{@usefulToolStyle}{%
linewidth=2pt,%
linecolor=usefultool-frame,%
roundcorner=4pt,%
backgroundcolor=usefultool-background,%
leftmargin=5pt,%
rightmargin=5pt,%
font={},%
nobreak=true%
}%
\mdtheorem[style=@usefulToolStyle]{@usefulTool}{\@usefulToolautorefname}%
%% Setting up the proper title
\xpatchcmd{\@usefulTool}{\refstepcounter}{\NR@gettitle{#1}\refstepcounter}{}{}%
%
%
%%% A Novice Hint is a box intended to give some useful information for novies
%%% or first-time readers.
\protected\gdef\noviceHint#1{%
\begin{center}%
\fcolorbox{novice-hint-frame}{novice-hint-background}{%
\parbox{0.9\linewidth}{%
\textbf{First Time Readers and Novices:}~#1
}}%
\end{center}%
}%
%
\ifIsBook\expandafter\@firstoftwo\else\expandafter\@secondoftwo\fi{%
\input{\bookbaseDir/styles/books/definitions.tex}%
}{}%
\ifIsSlides\expandafter\@firstoftwo\else\expandafter\@secondoftwo\fi{%
\input{\bookbaseDir/styles/slides/definitions.tex}%
}{}%
%
%
}{}%
\ifIsSlides\expandafter\@firstoftwo\else\expandafter\@secondoftwo\fi{%
%%
%% Commands for Definitions.
%%
\let\th@plain\relax%%%
\RequirePackage{ntheorem}%
\RequirePackage[framemethod=TikZ]{mdframed}%
%
\theorembodyfont{\normalfont}%
\newcommand{\definitionautorefname}{Definition}%%
%
\ifIsSlides\expandafter\@firstoftwo\else\expandafter\@secondoftwo\fi{%
\theoremstyle{nonumberplain}%
}{}%
%
%% Definitions
%% Definitions provide boxes in which we can define stuff.
\mdfdefinestyle{definitionStyle}{%
linewidth=2pt,%
linecolor=definition-frame,%
roundcorner=4pt,%
backgroundcolor=definition-background,%
leftmargin=5pt,%
rightmargin=5pt,%
font={},%
nobreak=true%
}%
%
%%% Define the definition environment based on the slides/book situation.
\ifIsSlides\expandafter\@firstoftwo\else\expandafter\@secondoftwo\fi{%
\mdtheorem[style=definitionStyle]{definition}{\definitionautorefname}%
}{%
\mdtheorem[style=definitionStyle]{definition}{\definitionautorefname}[chapter]%
}%
%
%% Setting up the proper title
\xpatchcmd{\definition}{\refstepcounter}{\NR@gettitle{#1}\refstepcounter}{}{}%
%
%
%%% Best Practices
%% The best practices suite offers the command \bestPractice that prints
%% information  about a best practice in a framed box.
%% The list of all best practices can be printed via \printBestPractices.
\newcommand{\@bestPracticeautorefname}{Best Practice}%
\mdfdefinestyle{@bestPracticeStyle}{%
linewidth=2pt,%
linecolor=bestpractice-frame,%
roundcorner=4pt,%
backgroundcolor=bestpractice-background,%
leftmargin=5pt,%
rightmargin=5pt,%
font={},%
nobreak=true%
}%
\mdtheorem[style=@bestPracticeStyle]{@bestPractice}{\@bestPracticeautorefname}%
%% Setting up the proper title
\xpatchcmd{\@bestPractice}{\refstepcounter}{\NR@gettitle{#1}\refstepcounter}{}{}%
%
%
%%% Useful Tools
%% The useful tools command suite offers the command \usefulTool that
%% prints information about a useful tool in a framed box.
%% The list of all useful tools can be printed via \printUsefulTools.
\newcommand{\@usefulToolautorefname}{Useful Tool}%
\mdfdefinestyle{@usefulToolStyle}{%
linewidth=2pt,%
linecolor=usefultool-frame,%
roundcorner=4pt,%
backgroundcolor=usefultool-background,%
leftmargin=5pt,%
rightmargin=5pt,%
font={},%
nobreak=true%
}%
\mdtheorem[style=@usefulToolStyle]{@usefulTool}{\@usefulToolautorefname}%
%% Setting up the proper title
\xpatchcmd{\@usefulTool}{\refstepcounter}{\NR@gettitle{#1}\refstepcounter}{}{}%
%
%
%%% A Novice Hint is a box intended to give some useful information for novies
%%% or first-time readers.
\protected\gdef\noviceHint#1{%
\begin{center}%
\fcolorbox{novice-hint-frame}{novice-hint-background}{%
\parbox{0.9\linewidth}{%
\textbf{First Time Readers and Novices:}~#1
}}%
\end{center}%
}%
%
\ifIsBook\expandafter\@firstoftwo\else\expandafter\@secondoftwo\fi{%
\input{\bookbaseDir/styles/books/definitions.tex}%
}{}%
\ifIsSlides\expandafter\@firstoftwo\else\expandafter\@secondoftwo\fi{%
\input{\bookbaseDir/styles/slides/definitions.tex}%
}{}%
%
%
}{}%
%
%
}{}%
%
%
}{}%
\ifIsSlides\expandafter\@firstoftwo\else\expandafter\@secondoftwo\fi{%
%%
%% Commands for Definitions.
%%
\let\th@plain\relax%%%
\RequirePackage{ntheorem}%
\RequirePackage[framemethod=TikZ]{mdframed}%
%
\theorembodyfont{\normalfont}%
\newcommand{\definitionautorefname}{Definition}%%
%
\ifIsSlides\expandafter\@firstoftwo\else\expandafter\@secondoftwo\fi{%
\theoremstyle{nonumberplain}%
}{}%
%
%% Definitions
%% Definitions provide boxes in which we can define stuff.
\mdfdefinestyle{definitionStyle}{%
linewidth=2pt,%
linecolor=definition-frame,%
roundcorner=4pt,%
backgroundcolor=definition-background,%
leftmargin=5pt,%
rightmargin=5pt,%
font={},%
nobreak=true%
}%
%
%%% Define the definition environment based on the slides/book situation.
\ifIsSlides\expandafter\@firstoftwo\else\expandafter\@secondoftwo\fi{%
\mdtheorem[style=definitionStyle]{definition}{\definitionautorefname}%
}{%
\mdtheorem[style=definitionStyle]{definition}{\definitionautorefname}[chapter]%
}%
%
%% Setting up the proper title
\xpatchcmd{\definition}{\refstepcounter}{\NR@gettitle{#1}\refstepcounter}{}{}%
%
%
%%% Best Practices
%% The best practices suite offers the command \bestPractice that prints
%% information  about a best practice in a framed box.
%% The list of all best practices can be printed via \printBestPractices.
\newcommand{\@bestPracticeautorefname}{Best Practice}%
\mdfdefinestyle{@bestPracticeStyle}{%
linewidth=2pt,%
linecolor=bestpractice-frame,%
roundcorner=4pt,%
backgroundcolor=bestpractice-background,%
leftmargin=5pt,%
rightmargin=5pt,%
font={},%
nobreak=true%
}%
\mdtheorem[style=@bestPracticeStyle]{@bestPractice}{\@bestPracticeautorefname}%
%% Setting up the proper title
\xpatchcmd{\@bestPractice}{\refstepcounter}{\NR@gettitle{#1}\refstepcounter}{}{}%
%
%
%%% Useful Tools
%% The useful tools command suite offers the command \usefulTool that
%% prints information about a useful tool in a framed box.
%% The list of all useful tools can be printed via \printUsefulTools.
\newcommand{\@usefulToolautorefname}{Useful Tool}%
\mdfdefinestyle{@usefulToolStyle}{%
linewidth=2pt,%
linecolor=usefultool-frame,%
roundcorner=4pt,%
backgroundcolor=usefultool-background,%
leftmargin=5pt,%
rightmargin=5pt,%
font={},%
nobreak=true%
}%
\mdtheorem[style=@usefulToolStyle]{@usefulTool}{\@usefulToolautorefname}%
%% Setting up the proper title
\xpatchcmd{\@usefulTool}{\refstepcounter}{\NR@gettitle{#1}\refstepcounter}{}{}%
%
%
%%% A Novice Hint is a box intended to give some useful information for novies
%%% or first-time readers.
\protected\gdef\noviceHint#1{%
\begin{center}%
\fcolorbox{novice-hint-frame}{novice-hint-background}{%
\parbox{0.9\linewidth}{%
\textbf{First Time Readers and Novices:}~#1
}}%
\end{center}%
}%
%
\ifIsBook\expandafter\@firstoftwo\else\expandafter\@secondoftwo\fi{%
%%
%% Commands for Definitions.
%%
\let\th@plain\relax%%%
\RequirePackage{ntheorem}%
\RequirePackage[framemethod=TikZ]{mdframed}%
%
\theorembodyfont{\normalfont}%
\newcommand{\definitionautorefname}{Definition}%%
%
\ifIsSlides\expandafter\@firstoftwo\else\expandafter\@secondoftwo\fi{%
\theoremstyle{nonumberplain}%
}{}%
%
%% Definitions
%% Definitions provide boxes in which we can define stuff.
\mdfdefinestyle{definitionStyle}{%
linewidth=2pt,%
linecolor=definition-frame,%
roundcorner=4pt,%
backgroundcolor=definition-background,%
leftmargin=5pt,%
rightmargin=5pt,%
font={},%
nobreak=true%
}%
%
%%% Define the definition environment based on the slides/book situation.
\ifIsSlides\expandafter\@firstoftwo\else\expandafter\@secondoftwo\fi{%
\mdtheorem[style=definitionStyle]{definition}{\definitionautorefname}%
}{%
\mdtheorem[style=definitionStyle]{definition}{\definitionautorefname}[chapter]%
}%
%
%% Setting up the proper title
\xpatchcmd{\definition}{\refstepcounter}{\NR@gettitle{#1}\refstepcounter}{}{}%
%
%
%%% Best Practices
%% The best practices suite offers the command \bestPractice that prints
%% information  about a best practice in a framed box.
%% The list of all best practices can be printed via \printBestPractices.
\newcommand{\@bestPracticeautorefname}{Best Practice}%
\mdfdefinestyle{@bestPracticeStyle}{%
linewidth=2pt,%
linecolor=bestpractice-frame,%
roundcorner=4pt,%
backgroundcolor=bestpractice-background,%
leftmargin=5pt,%
rightmargin=5pt,%
font={},%
nobreak=true%
}%
\mdtheorem[style=@bestPracticeStyle]{@bestPractice}{\@bestPracticeautorefname}%
%% Setting up the proper title
\xpatchcmd{\@bestPractice}{\refstepcounter}{\NR@gettitle{#1}\refstepcounter}{}{}%
%
%
%%% Useful Tools
%% The useful tools command suite offers the command \usefulTool that
%% prints information about a useful tool in a framed box.
%% The list of all useful tools can be printed via \printUsefulTools.
\newcommand{\@usefulToolautorefname}{Useful Tool}%
\mdfdefinestyle{@usefulToolStyle}{%
linewidth=2pt,%
linecolor=usefultool-frame,%
roundcorner=4pt,%
backgroundcolor=usefultool-background,%
leftmargin=5pt,%
rightmargin=5pt,%
font={},%
nobreak=true%
}%
\mdtheorem[style=@usefulToolStyle]{@usefulTool}{\@usefulToolautorefname}%
%% Setting up the proper title
\xpatchcmd{\@usefulTool}{\refstepcounter}{\NR@gettitle{#1}\refstepcounter}{}{}%
%
%
%%% A Novice Hint is a box intended to give some useful information for novies
%%% or first-time readers.
\protected\gdef\noviceHint#1{%
\begin{center}%
\fcolorbox{novice-hint-frame}{novice-hint-background}{%
\parbox{0.9\linewidth}{%
\textbf{First Time Readers and Novices:}~#1
}}%
\end{center}%
}%
%
\ifIsBook\expandafter\@firstoftwo\else\expandafter\@secondoftwo\fi{%
%%
%% Commands for Definitions.
%%
\let\th@plain\relax%%%
\RequirePackage{ntheorem}%
\RequirePackage[framemethod=TikZ]{mdframed}%
%
\theorembodyfont{\normalfont}%
\newcommand{\definitionautorefname}{Definition}%%
%
\ifIsSlides\expandafter\@firstoftwo\else\expandafter\@secondoftwo\fi{%
\theoremstyle{nonumberplain}%
}{}%
%
%% Definitions
%% Definitions provide boxes in which we can define stuff.
\mdfdefinestyle{definitionStyle}{%
linewidth=2pt,%
linecolor=definition-frame,%
roundcorner=4pt,%
backgroundcolor=definition-background,%
leftmargin=5pt,%
rightmargin=5pt,%
font={},%
nobreak=true%
}%
%
%%% Define the definition environment based on the slides/book situation.
\ifIsSlides\expandafter\@firstoftwo\else\expandafter\@secondoftwo\fi{%
\mdtheorem[style=definitionStyle]{definition}{\definitionautorefname}%
}{%
\mdtheorem[style=definitionStyle]{definition}{\definitionautorefname}[chapter]%
}%
%
%% Setting up the proper title
\xpatchcmd{\definition}{\refstepcounter}{\NR@gettitle{#1}\refstepcounter}{}{}%
%
%
%%% Best Practices
%% The best practices suite offers the command \bestPractice that prints
%% information  about a best practice in a framed box.
%% The list of all best practices can be printed via \printBestPractices.
\newcommand{\@bestPracticeautorefname}{Best Practice}%
\mdfdefinestyle{@bestPracticeStyle}{%
linewidth=2pt,%
linecolor=bestpractice-frame,%
roundcorner=4pt,%
backgroundcolor=bestpractice-background,%
leftmargin=5pt,%
rightmargin=5pt,%
font={},%
nobreak=true%
}%
\mdtheorem[style=@bestPracticeStyle]{@bestPractice}{\@bestPracticeautorefname}%
%% Setting up the proper title
\xpatchcmd{\@bestPractice}{\refstepcounter}{\NR@gettitle{#1}\refstepcounter}{}{}%
%
%
%%% Useful Tools
%% The useful tools command suite offers the command \usefulTool that
%% prints information about a useful tool in a framed box.
%% The list of all useful tools can be printed via \printUsefulTools.
\newcommand{\@usefulToolautorefname}{Useful Tool}%
\mdfdefinestyle{@usefulToolStyle}{%
linewidth=2pt,%
linecolor=usefultool-frame,%
roundcorner=4pt,%
backgroundcolor=usefultool-background,%
leftmargin=5pt,%
rightmargin=5pt,%
font={},%
nobreak=true%
}%
\mdtheorem[style=@usefulToolStyle]{@usefulTool}{\@usefulToolautorefname}%
%% Setting up the proper title
\xpatchcmd{\@usefulTool}{\refstepcounter}{\NR@gettitle{#1}\refstepcounter}{}{}%
%
%
%%% A Novice Hint is a box intended to give some useful information for novies
%%% or first-time readers.
\protected\gdef\noviceHint#1{%
\begin{center}%
\fcolorbox{novice-hint-frame}{novice-hint-background}{%
\parbox{0.9\linewidth}{%
\textbf{First Time Readers and Novices:}~#1
}}%
\end{center}%
}%
%
\ifIsBook\expandafter\@firstoftwo\else\expandafter\@secondoftwo\fi{%
\input{\bookbaseDir/styles/books/definitions.tex}%
}{}%
\ifIsSlides\expandafter\@firstoftwo\else\expandafter\@secondoftwo\fi{%
\input{\bookbaseDir/styles/slides/definitions.tex}%
}{}%
%
%
}{}%
\ifIsSlides\expandafter\@firstoftwo\else\expandafter\@secondoftwo\fi{%
%%
%% Commands for Definitions.
%%
\let\th@plain\relax%%%
\RequirePackage{ntheorem}%
\RequirePackage[framemethod=TikZ]{mdframed}%
%
\theorembodyfont{\normalfont}%
\newcommand{\definitionautorefname}{Definition}%%
%
\ifIsSlides\expandafter\@firstoftwo\else\expandafter\@secondoftwo\fi{%
\theoremstyle{nonumberplain}%
}{}%
%
%% Definitions
%% Definitions provide boxes in which we can define stuff.
\mdfdefinestyle{definitionStyle}{%
linewidth=2pt,%
linecolor=definition-frame,%
roundcorner=4pt,%
backgroundcolor=definition-background,%
leftmargin=5pt,%
rightmargin=5pt,%
font={},%
nobreak=true%
}%
%
%%% Define the definition environment based on the slides/book situation.
\ifIsSlides\expandafter\@firstoftwo\else\expandafter\@secondoftwo\fi{%
\mdtheorem[style=definitionStyle]{definition}{\definitionautorefname}%
}{%
\mdtheorem[style=definitionStyle]{definition}{\definitionautorefname}[chapter]%
}%
%
%% Setting up the proper title
\xpatchcmd{\definition}{\refstepcounter}{\NR@gettitle{#1}\refstepcounter}{}{}%
%
%
%%% Best Practices
%% The best practices suite offers the command \bestPractice that prints
%% information  about a best practice in a framed box.
%% The list of all best practices can be printed via \printBestPractices.
\newcommand{\@bestPracticeautorefname}{Best Practice}%
\mdfdefinestyle{@bestPracticeStyle}{%
linewidth=2pt,%
linecolor=bestpractice-frame,%
roundcorner=4pt,%
backgroundcolor=bestpractice-background,%
leftmargin=5pt,%
rightmargin=5pt,%
font={},%
nobreak=true%
}%
\mdtheorem[style=@bestPracticeStyle]{@bestPractice}{\@bestPracticeautorefname}%
%% Setting up the proper title
\xpatchcmd{\@bestPractice}{\refstepcounter}{\NR@gettitle{#1}\refstepcounter}{}{}%
%
%
%%% Useful Tools
%% The useful tools command suite offers the command \usefulTool that
%% prints information about a useful tool in a framed box.
%% The list of all useful tools can be printed via \printUsefulTools.
\newcommand{\@usefulToolautorefname}{Useful Tool}%
\mdfdefinestyle{@usefulToolStyle}{%
linewidth=2pt,%
linecolor=usefultool-frame,%
roundcorner=4pt,%
backgroundcolor=usefultool-background,%
leftmargin=5pt,%
rightmargin=5pt,%
font={},%
nobreak=true%
}%
\mdtheorem[style=@usefulToolStyle]{@usefulTool}{\@usefulToolautorefname}%
%% Setting up the proper title
\xpatchcmd{\@usefulTool}{\refstepcounter}{\NR@gettitle{#1}\refstepcounter}{}{}%
%
%
%%% A Novice Hint is a box intended to give some useful information for novies
%%% or first-time readers.
\protected\gdef\noviceHint#1{%
\begin{center}%
\fcolorbox{novice-hint-frame}{novice-hint-background}{%
\parbox{0.9\linewidth}{%
\textbf{First Time Readers and Novices:}~#1
}}%
\end{center}%
}%
%
\ifIsBook\expandafter\@firstoftwo\else\expandafter\@secondoftwo\fi{%
\input{\bookbaseDir/styles/books/definitions.tex}%
}{}%
\ifIsSlides\expandafter\@firstoftwo\else\expandafter\@secondoftwo\fi{%
\input{\bookbaseDir/styles/slides/definitions.tex}%
}{}%
%
%
}{}%
%
%
}{}%
\ifIsSlides\expandafter\@firstoftwo\else\expandafter\@secondoftwo\fi{%
%%
%% Commands for Definitions.
%%
\let\th@plain\relax%%%
\RequirePackage{ntheorem}%
\RequirePackage[framemethod=TikZ]{mdframed}%
%
\theorembodyfont{\normalfont}%
\newcommand{\definitionautorefname}{Definition}%%
%
\ifIsSlides\expandafter\@firstoftwo\else\expandafter\@secondoftwo\fi{%
\theoremstyle{nonumberplain}%
}{}%
%
%% Definitions
%% Definitions provide boxes in which we can define stuff.
\mdfdefinestyle{definitionStyle}{%
linewidth=2pt,%
linecolor=definition-frame,%
roundcorner=4pt,%
backgroundcolor=definition-background,%
leftmargin=5pt,%
rightmargin=5pt,%
font={},%
nobreak=true%
}%
%
%%% Define the definition environment based on the slides/book situation.
\ifIsSlides\expandafter\@firstoftwo\else\expandafter\@secondoftwo\fi{%
\mdtheorem[style=definitionStyle]{definition}{\definitionautorefname}%
}{%
\mdtheorem[style=definitionStyle]{definition}{\definitionautorefname}[chapter]%
}%
%
%% Setting up the proper title
\xpatchcmd{\definition}{\refstepcounter}{\NR@gettitle{#1}\refstepcounter}{}{}%
%
%
%%% Best Practices
%% The best practices suite offers the command \bestPractice that prints
%% information  about a best practice in a framed box.
%% The list of all best practices can be printed via \printBestPractices.
\newcommand{\@bestPracticeautorefname}{Best Practice}%
\mdfdefinestyle{@bestPracticeStyle}{%
linewidth=2pt,%
linecolor=bestpractice-frame,%
roundcorner=4pt,%
backgroundcolor=bestpractice-background,%
leftmargin=5pt,%
rightmargin=5pt,%
font={},%
nobreak=true%
}%
\mdtheorem[style=@bestPracticeStyle]{@bestPractice}{\@bestPracticeautorefname}%
%% Setting up the proper title
\xpatchcmd{\@bestPractice}{\refstepcounter}{\NR@gettitle{#1}\refstepcounter}{}{}%
%
%
%%% Useful Tools
%% The useful tools command suite offers the command \usefulTool that
%% prints information about a useful tool in a framed box.
%% The list of all useful tools can be printed via \printUsefulTools.
\newcommand{\@usefulToolautorefname}{Useful Tool}%
\mdfdefinestyle{@usefulToolStyle}{%
linewidth=2pt,%
linecolor=usefultool-frame,%
roundcorner=4pt,%
backgroundcolor=usefultool-background,%
leftmargin=5pt,%
rightmargin=5pt,%
font={},%
nobreak=true%
}%
\mdtheorem[style=@usefulToolStyle]{@usefulTool}{\@usefulToolautorefname}%
%% Setting up the proper title
\xpatchcmd{\@usefulTool}{\refstepcounter}{\NR@gettitle{#1}\refstepcounter}{}{}%
%
%
%%% A Novice Hint is a box intended to give some useful information for novies
%%% or first-time readers.
\protected\gdef\noviceHint#1{%
\begin{center}%
\fcolorbox{novice-hint-frame}{novice-hint-background}{%
\parbox{0.9\linewidth}{%
\textbf{First Time Readers and Novices:}~#1
}}%
\end{center}%
}%
%
\ifIsBook\expandafter\@firstoftwo\else\expandafter\@secondoftwo\fi{%
%%
%% Commands for Definitions.
%%
\let\th@plain\relax%%%
\RequirePackage{ntheorem}%
\RequirePackage[framemethod=TikZ]{mdframed}%
%
\theorembodyfont{\normalfont}%
\newcommand{\definitionautorefname}{Definition}%%
%
\ifIsSlides\expandafter\@firstoftwo\else\expandafter\@secondoftwo\fi{%
\theoremstyle{nonumberplain}%
}{}%
%
%% Definitions
%% Definitions provide boxes in which we can define stuff.
\mdfdefinestyle{definitionStyle}{%
linewidth=2pt,%
linecolor=definition-frame,%
roundcorner=4pt,%
backgroundcolor=definition-background,%
leftmargin=5pt,%
rightmargin=5pt,%
font={},%
nobreak=true%
}%
%
%%% Define the definition environment based on the slides/book situation.
\ifIsSlides\expandafter\@firstoftwo\else\expandafter\@secondoftwo\fi{%
\mdtheorem[style=definitionStyle]{definition}{\definitionautorefname}%
}{%
\mdtheorem[style=definitionStyle]{definition}{\definitionautorefname}[chapter]%
}%
%
%% Setting up the proper title
\xpatchcmd{\definition}{\refstepcounter}{\NR@gettitle{#1}\refstepcounter}{}{}%
%
%
%%% Best Practices
%% The best practices suite offers the command \bestPractice that prints
%% information  about a best practice in a framed box.
%% The list of all best practices can be printed via \printBestPractices.
\newcommand{\@bestPracticeautorefname}{Best Practice}%
\mdfdefinestyle{@bestPracticeStyle}{%
linewidth=2pt,%
linecolor=bestpractice-frame,%
roundcorner=4pt,%
backgroundcolor=bestpractice-background,%
leftmargin=5pt,%
rightmargin=5pt,%
font={},%
nobreak=true%
}%
\mdtheorem[style=@bestPracticeStyle]{@bestPractice}{\@bestPracticeautorefname}%
%% Setting up the proper title
\xpatchcmd{\@bestPractice}{\refstepcounter}{\NR@gettitle{#1}\refstepcounter}{}{}%
%
%
%%% Useful Tools
%% The useful tools command suite offers the command \usefulTool that
%% prints information about a useful tool in a framed box.
%% The list of all useful tools can be printed via \printUsefulTools.
\newcommand{\@usefulToolautorefname}{Useful Tool}%
\mdfdefinestyle{@usefulToolStyle}{%
linewidth=2pt,%
linecolor=usefultool-frame,%
roundcorner=4pt,%
backgroundcolor=usefultool-background,%
leftmargin=5pt,%
rightmargin=5pt,%
font={},%
nobreak=true%
}%
\mdtheorem[style=@usefulToolStyle]{@usefulTool}{\@usefulToolautorefname}%
%% Setting up the proper title
\xpatchcmd{\@usefulTool}{\refstepcounter}{\NR@gettitle{#1}\refstepcounter}{}{}%
%
%
%%% A Novice Hint is a box intended to give some useful information for novies
%%% or first-time readers.
\protected\gdef\noviceHint#1{%
\begin{center}%
\fcolorbox{novice-hint-frame}{novice-hint-background}{%
\parbox{0.9\linewidth}{%
\textbf{First Time Readers and Novices:}~#1
}}%
\end{center}%
}%
%
\ifIsBook\expandafter\@firstoftwo\else\expandafter\@secondoftwo\fi{%
\input{\bookbaseDir/styles/books/definitions.tex}%
}{}%
\ifIsSlides\expandafter\@firstoftwo\else\expandafter\@secondoftwo\fi{%
\input{\bookbaseDir/styles/slides/definitions.tex}%
}{}%
%
%
}{}%
\ifIsSlides\expandafter\@firstoftwo\else\expandafter\@secondoftwo\fi{%
%%
%% Commands for Definitions.
%%
\let\th@plain\relax%%%
\RequirePackage{ntheorem}%
\RequirePackage[framemethod=TikZ]{mdframed}%
%
\theorembodyfont{\normalfont}%
\newcommand{\definitionautorefname}{Definition}%%
%
\ifIsSlides\expandafter\@firstoftwo\else\expandafter\@secondoftwo\fi{%
\theoremstyle{nonumberplain}%
}{}%
%
%% Definitions
%% Definitions provide boxes in which we can define stuff.
\mdfdefinestyle{definitionStyle}{%
linewidth=2pt,%
linecolor=definition-frame,%
roundcorner=4pt,%
backgroundcolor=definition-background,%
leftmargin=5pt,%
rightmargin=5pt,%
font={},%
nobreak=true%
}%
%
%%% Define the definition environment based on the slides/book situation.
\ifIsSlides\expandafter\@firstoftwo\else\expandafter\@secondoftwo\fi{%
\mdtheorem[style=definitionStyle]{definition}{\definitionautorefname}%
}{%
\mdtheorem[style=definitionStyle]{definition}{\definitionautorefname}[chapter]%
}%
%
%% Setting up the proper title
\xpatchcmd{\definition}{\refstepcounter}{\NR@gettitle{#1}\refstepcounter}{}{}%
%
%
%%% Best Practices
%% The best practices suite offers the command \bestPractice that prints
%% information  about a best practice in a framed box.
%% The list of all best practices can be printed via \printBestPractices.
\newcommand{\@bestPracticeautorefname}{Best Practice}%
\mdfdefinestyle{@bestPracticeStyle}{%
linewidth=2pt,%
linecolor=bestpractice-frame,%
roundcorner=4pt,%
backgroundcolor=bestpractice-background,%
leftmargin=5pt,%
rightmargin=5pt,%
font={},%
nobreak=true%
}%
\mdtheorem[style=@bestPracticeStyle]{@bestPractice}{\@bestPracticeautorefname}%
%% Setting up the proper title
\xpatchcmd{\@bestPractice}{\refstepcounter}{\NR@gettitle{#1}\refstepcounter}{}{}%
%
%
%%% Useful Tools
%% The useful tools command suite offers the command \usefulTool that
%% prints information about a useful tool in a framed box.
%% The list of all useful tools can be printed via \printUsefulTools.
\newcommand{\@usefulToolautorefname}{Useful Tool}%
\mdfdefinestyle{@usefulToolStyle}{%
linewidth=2pt,%
linecolor=usefultool-frame,%
roundcorner=4pt,%
backgroundcolor=usefultool-background,%
leftmargin=5pt,%
rightmargin=5pt,%
font={},%
nobreak=true%
}%
\mdtheorem[style=@usefulToolStyle]{@usefulTool}{\@usefulToolautorefname}%
%% Setting up the proper title
\xpatchcmd{\@usefulTool}{\refstepcounter}{\NR@gettitle{#1}\refstepcounter}{}{}%
%
%
%%% A Novice Hint is a box intended to give some useful information for novies
%%% or first-time readers.
\protected\gdef\noviceHint#1{%
\begin{center}%
\fcolorbox{novice-hint-frame}{novice-hint-background}{%
\parbox{0.9\linewidth}{%
\textbf{First Time Readers and Novices:}~#1
}}%
\end{center}%
}%
%
\ifIsBook\expandafter\@firstoftwo\else\expandafter\@secondoftwo\fi{%
\input{\bookbaseDir/styles/books/definitions.tex}%
}{}%
\ifIsSlides\expandafter\@firstoftwo\else\expandafter\@secondoftwo\fi{%
\input{\bookbaseDir/styles/slides/definitions.tex}%
}{}%
%
%
}{}%
%
%
}{}%
%
%
}{}%
%

%%% Mathematical Notation
%%%
\protected\gdef\decSep{\hspace*{0.175em}}%
\protected\gdef\mathSpace#1{\ensuremath{\mathbb{#1}}}%
%
%
%
\newFunc{arithMean}{\ensuremath{\operatorname{mean}}}{mean(A)}{A}{%
The \emph{arithmetic mean}~\arithMeanb{A} is an estimate of the expected value of a distribution from which a data sample was, well, sampled. %
Its is computed on data sample~$A=(\arrayIndex{a}{0},\arrayIndex{a}{1},\dots,\arrayIndex{a}{n-1})$ as the sum of all $n$ elements~$\arrayIndex{a}{i}$ in the sample data~$A$ divided by the total number~$n$ of values, i.e., $\arithMeanb{A} = \frac{1}{n} \sum_{i=0}^{n-1} \arrayIndex{a}{i}$%
}%
%
%
%% Access an element in an array or list
%% #1 the array
%% #2 the element to access
\protected\gdef\arrayIndex#1#2{\ensuremath{#1_{#2}}}%
%
%
\newFunc{attrDomain}{\ensuremath{\operatorname{dom}}}{dom(a)}{a}{%
the domain of an attribute~$a$\optionalRef{def:attributeDomain}{, see \cref{def:attributeDomain}}}%
%
%
\newFunc{bigO}{\ensuremath{\mathcal{O}}}{O(g(x))}{g(x)}{%
\mbox{If $f(x)=\bigOb{g(x)}$,} then there exist positive numbers~$x_0\in\realNumbersP$ and~$c\in\realNumbersP$ such that~\mbox{$f(x)\leq c*g(x)\forall x\geq x_0$}~\cite{B1894DAZDVPB,L1909HDLVDVDP}. %
In other words, \bigOb{g(x)}~describes an upper bound for function growth}%
%
%
\protected\gdef\factorial#1{\ensuremath{#1\gls{factorialMark}}}%
\newglossaryentry{factorialMark}{%
type={symbols},
name={$i!$},%
text={!},%
sort={!},%
description={\sloppy%
The factorial~\factorial{a} of a natural number~$a\in\naturalNumbersO$ is the product of all positive natural numbers less than or equal to~$a$, i.e., $\factorial{a}=1*2*3*4*\dots*(a-1)*a$~\cite{D1991TEHOTFF,CB2022FBDOTFF,L2015ANKOFF}.\optionalRef{eq:factorial}{ %
See also \cref{eq:factorial} and \cref{lst:functions:def_factorial}}%
}}%
%
%
\newFunc{geoMean}{\ensuremath{\operatorname{geom}}}{geom(A)}{A}{%
The \emph{geometric mean}~\geoMeanb{A} is the $n$\textsuperscript{th} root of the product of $n$ \emph{positive} values in a dataset~$A=(\arrayIndex{a}{0},\arrayIndex{a}{1},\dots,\arrayIndex{a}{n-1})$ with $\arrayIndex{a}{i}>0$ for all~$i\in\intRange{0}{n}$, i.e., $\geoMeanb{A} = \sqrt[n]{\prod_{i=0}^{n-1} \arrayIndex{a}{i}}=\exp{\left(\frac{1}{n} \sum_{i=0}^{n-1} \log{\arrayIndex{a}{i}} \right)}$%
}%
%
%
\newSymbol{integerNumbers}{\ensuremath{\mathSpace{Z}}}{Z}{%
the set of the integers numbers including positive and negative numbers and~0, i.e., {\dots}, -3, -2, -1, 0, 1, 2, 3, {\dots}, and so on. %
It holds that~$\integerNumbers\subset\realNumbers$%
}%
%
%
\protected\gdef\intRange#1#2{\ensuremath{#1\gls{intSetDots}#2}}%
\newglossaryentry{intSetDots}{%
type={symbols},
name={$i..j$},%
text={..},%
sort={..},%
description={\sloppy%
with $i,j\in\integerNumbers$ and $i\leq j$ is the set that contains all integer numbers in the inclusive range from~$i$ to~$j$. %
For example, \mbox{\intRange{5}{9}}~is equivalent to~\mbox{$\{5, 6, 7, 8, 9\}$}}%
}%
%
%
\newFunc{median}{\ensuremath{\operatorname{median}}}{median(A)}{A}{%
The \emph{median}~\medianb{A} is the value separating the bigger-valued half from the smaller-valued half of a data sample or distribution. %
Its estimate is the value right in the middle of a \emph{sorted} data sample~$A=(\arrayIndex{a}{0},\arrayIndex{a}{1}, \dots, \arrayIndex{a}{n-1})$ where $\arrayIndex{a}{i-1}\leq \arrayIndex{a}{i} \; \forall i \in 1\dots (n-1)$ with an odd number of elements and the \arithMean\ of the two values in the middle if $n$ is even. %
In other words, $\medianb{A} = \arrayIndex{a}{\frac{n-1}{2}}$ if $n$ is odd and $\frac{1}{2}\left(\mathNoTopSpacing{\arrayIndex{a}{\frac{n}{2}-1} + \arrayIndex{a}{\frac{n}{2}}}\right)$ otherwise, i.e., if $n$ is even%
}%
%
%
\newSymbol{naturalNumbersO}{\ensuremath{\mathSpace{N}_1}}{N1}{%
the set of the natural numbers \emph{excluding}~0, i.e., 1, 2, 3, 4, and so on. %
It holds that~$\naturalNumbersO\subset\integerNumbers$%
}%
%
%
\newSymbol{naturalNumbersZ}{\ensuremath{\mathSpace{N}_0}}{N0}{%
the set of the natural numbers \emph{including}~0, i.e., 0, 1, 2, 3, and so on. %
It holds that~$\naturalNumbersZ\subset\integerNumbers$%
}%
%
%
\protected\gdef\@Pprefix{\ensuremath{\mathcal{P}}}%
\protected\gdef\@NPprefix{\ensuremath{\mathcal{NP}}}%
%
\newSymbol{classNP}{\@NPprefix}{NP}{%
\ensuremath{\mathcal{NP}}~is the class of computational problems that can be solved in polynomial time by a non-deterministic machine and can be verified in polynomial time by a deterministic machine~(such as a normal computer)~\cite{G2022AGITNC}.%
}%
%
\newSymbol{npComplete}{\@NPprefix\nobreakdashes-complete}{NP-complete}{%
A decision problem is \gls{classNP}\nobreakdashes-complete if it is in \@NPprefix\ and all problems in \@NPprefix\ are reducible to it in polynomial time~\cite{RV2008NCPAPM,G2022AGITNC}. %
A problem is \npComplete\ if it is \npHard\ and if it is in \@NPprefix.%
}%
%
\newSymbol{npHard}{\@NPprefix\nobreakdashes-hard}{NP-hard}{%
Algorithms that guarantee to find the correct solutions of \gls{classNP}\nobreakdashes-hard problems~\cite{LLRKS1993SASAAC,CPW1998AROMSCAAA,C1971TCOTPP} need a runtime that is exponential in the problem scale in the worst case. %
A problem is \npHard\ if all problems in \@NPprefix\ are reducible to it in polynomial time~\cite{G2022AGITNC}.%
}%
%
%
\newSymbol{numberE}{\ensuremath{e}}{e}{%
is Euler's number~\cite{E1737DFCD}, the base of the natural logarithm. %
$\numberE\in\realNumbers$ is an irrational and transcendental number~\cite{APM1991AAAFI:TOEAP,F2011TTOEAP}, which is approximately~$\numberE\approx2.718\decSep281\decSep828\decSep459\decSep045\decSep235\decSep360$. %
In \python, it is provided by the \pythonilIdx{math} module as constant \pythonilIdx{e} with value~\pythonilIdx{2.718281828459045}}%
%
%
\newSymbol{numberGoldenRatio}{\ensuremath{\phi}}{$\phi$}{%
The golden ratio~(or golden section)~\numberGoldenRatio\ is the irrational number~$\frac{1+\sqrt{5}}{2}$. %
It is the ratio of a line segment cut into two pieces of different lengths such that the ratio of the whole segment to that of the longer segment is equal to the ratio of the longer segment to the shorter segment~\cite{EOEBEB:GR,EHF2008EEOGTGOJLH11FEEEELIEILHIATBG11EAPWMETBFR}. %
The golden ratio is approximately~$\phi\approx1.618\decSep033\decSep988\decSep749\decSep894\decSep848\decSep204\decSep586\decSep834$~\cite{S2024DEOGRPOT}. %
Represented as \pythonil{float} in \python, its value is~\pythonil{1.618033988749895}}%
%
%
\newSymbol{numberPi}{\ensuremath{\pi}}{$\pi$}{%
is the ratio of the circumference~$U$ of a circle and its diameter~$d$, i.e., $\numberPi=U/d$. %
$\numberPi\in\realNumbers$ is an irrational and transcendental number~\cite{N1939TTOP,APM1991AAAFI:TOEAP,F2011TTOEAP}, which is approximately~$\numberPi\approx3.141\decSep592\decSep653\decSep589\decSep793\decSep238\decSep462\decSep643$. %
In \python, it is provided by the \pythonilIdx{math} module as constant \pythonilIdx{pi} with value~\pythonilIdx{3.141592653589793}}%
%
%
\newSymbol{rationalNumbers}{\mathSpace{Q}}{Q}{%
the set of the rational numbers, i.e., the set of all numbers that can be the result of~$\frac{a}{b}$ with~$a,b\in\integerNumbers$ and~$b\neq0$. %
$a$~is called the \pgls{numerator} and $b$~is called the \pgls{denominator}. %
It holds that~$\integerNumbers\subset\rationalNumbers$ and $\rationalNumbers\subset\realNumbers$%
}%
%
%
\newSymbol{realNumbers}{\mathSpace{R}}{R}{the set of the real numbers}%
%
%
\newSymbol{realNumbersP}{\ensuremath{\mathSpace{R}^+}}{R+}{%
the set of the positive real numbers, i.e., $\realNumbersP=\{x\in\realNumbers:x>0\}$}%
%
%
\newFunc{relSchema}{\ensuremath{\operatorname{\Sigma}}}{Sigma(R)}{R}{%
the relation schema of relation~$R$~\cite{SS2005EIDDDFDB:SDWSD2}\optionalRef{def:relationSchema}{, see \cref{def:relationSchema}}}%
%
%
\protected\gdef\sampleQuantile#1#2#3{\ensuremath{\gls{sampleQuantileMark}^{#1}_{#2}(#3)}}%
\newglossaryentry{sampleQuantileMark}{%
type={symbols},
name={\ensuremath{\operatorname{quantile}^k_q(A)}},%
text={\ensuremath{\operatorname{quantile}}},%
sort={quantilekqA},%
description={%
The \emph{$q$\nobreakdashes-quantiles} are the cut points that divide a sorted data sample~$A=(\arrayIndex{a}{0},\arrayIndex{a}{1}, \dots, \arrayIndex{a}{n-1})$ where $\arrayIndex{a}{i-1}\leq \arrayIndex{a}{i} \; \forall i \in 1\dots (n-1)$ into $q$ equally-sized parts. %
\sampleQuantile{k}{q}{A}~be the $k$\textsuperscript{th}~$q$\nobreakdashes-quantile, with~$k\in 1\dots (q-1)$, i.e., there are $q-1$ of the $q$\nobreakdashes-quantiles. %
In the context of this book, define~$h=(n-1)\frac{k}{q}$. %
\sampleQuantile{k}{q}{A}~then can be computed as \arrayIndex{a}{h} if $h$~is integer, i.e.,~$h\in\integerNumbers$, and as ~$\arrayIndex{a}{\lfloor h\rfloor}+\left(h-\lfloor h\rfloor\right)*\left(\arrayIndex{a}{\lfloor h\rfloor+1}-\arrayIndex{a}{\lfloor h\rfloor}\right)$ otherwise. %
It holds that~$\sampleQuantile{2}{1}{A}=\medianb{A}$%
}}%
%
%
\newFunc{sampleStdDev}{\ensuremath{\operatorname{sd}}}{sd(A)}{A}{%
The statistical estimate~\sampleStdDevb{A} of the \emph{standard deviation} of a data sample~$A=(\arrayIndex{a}{0},\arrayIndex{a}{1}, \dots, \arrayIndex{a}{n-1})$ with $n$~observations is the square root of the estimated variance~$\sampleVarb{A}$, i.e., $\sampleStdDev{A} = \sqrt{\sampleVarb{A}}$%
}%
%
%
\newFunc{sampleVar}{\ensuremath{\operatorname{var}}}{var(A)}{A}{%
The \emph{variance} of a distribution is the expectation of the squared deviation of the underlying random variable from its mean. %
The variance~\sampleVarb{A} of a data sample~$A=(\arrayIndex{a}{0},\arrayIndex{a}{1}, \dots, \arrayIndex{a}{n-1})$ with $n$~observations can be estimated as~$\sampleVarb{A} = \frac{1}{n-1} \sum_{i=0}^{n-1} \left(\arrayIndex{a}{i} - \arithMeanb{A}\right)^2$%
}%

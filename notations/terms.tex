%% Access to Source Codes
%%
%
%
\newglossaryentry{bias}{%
name={bias},%
description={%
The bias is subtracted from the value stored in the \gls{exponent} field of a floating point number. %
This allows for representing both positive and negative exponents. %
In the 64~bit double precision IEEE~Standard 754 floating point number layout~\cite{IEEE2019ISFFPA,H1997IS7FPN}, the bias is~1023.%
\optionalRef{sec:howFloatingPointNumbersWork}{ See \cref{sec:howFloatingPointNumbersWork}.}%
}%
}%
%
%
\newglossaryentry{breakpoint}{%
name={breakpoint},%
description={%
A breakpoint is a mark in a line of code in an~\gls{ide} at which the \debugger\ will pause the execution of a program.%
}%
}%
%
%
\newglossaryentry{C}{%
text={\softwareStyle{C}},%
name={C},%
sort={C},%
description={%
is a programming language, which is very successful in system programming situations~\cite{ISOIEC98892017PLCWDOS,D2024MCFABAFITTCPL}.%
}%
}%
%
%
\newglossaryentry{client}{%
text={client},%
name={client},%
sort={client},%
description={%
In a \gls{clientServerArchitecture}, the \gls{client} is a device or process that requests a service from the \gls{server}. %
It initiates the communication with the \gls{server}, sends a request, and receives the response with the result of the request. %
Typical examples for \glspl{client} are web browsers in the internet as well as \glspl{client} for \glspl{dbms}, such as \psql.%
}%
}%
\protected\gdef\client{\pgls{client}}%
%
%
\newglossaryentry{clientServerArchitecture}{%
text={\gls{client}\nobreakdashes-\gls{server} architecture},%
name={client-server architecture},%
sort={client-server architecture},%
description={%
is a system design where a central \gls{server} receives requests from one or multiple \glspl{client}~\cite{RCKS2019PNP,B1996CSA,OHE1999CSSG,RF2020FOSAAEA,EOEBLGGCH2025CSA}. %
These requests and responses are usually sent over network connections. %
A typical example for such a system is the \gls{WWW}, where web \glspl{server} host websites and make them available to web browsers, the \glspl{client}. %
Another typical example is the structure of \gls{db} software, where a central \gls{server}, the \glslink{dbms}, offers access to the \gls{db} to the different \glspl{client}. %
Here, the \gls{client} can be some \gls{terminal} software shipping with the \glslink{dbms}, such as \psql, or the different applications that access the \glspl{db}.%
}%
}%
%
%
%
\newglossaryentry{docstring}{%
name={docstring},%
description={%
Docstrings are special string constants in \python\ that contain documentation for modules or functions~\cite{PEP257}. %
They must be delimited by~\pythonil{"""..."""}\pythonIdx{\textquotedbl\textquotedbl\textquotedbl\idxdots\textquotedbl\textquotedbl\textquotedbl}~\cite{PEP257,PEP8}.%
}%
}%
%
%
\newglossaryentry{denominator}{%
name={denominator},%
description={%
The number~$b$ of a fraction~$\frac{a}{b}\in\rationalNumbers$ is called the \emph{denominator}.}%
}%
%
%
\newglossaryentry{doctest}{%
name={doctest},%
description={%
\emph{doctests} are \pglspl{unitTest} in the form of as small pieces of code in the \glspl{docstring} that look like interactive \python\ sessions.
The first line of a statement in such a \python\ snippet is indented with \python{>>>}\pythonIdx{>\strut>\strut>} and the following lines by \pythonil{...}\pythonIdx{\idxdots}.
These snippets can be executed by modules like \pythonilIdx{doctest}~\cite{PSF2024DTIPE} or tools such as \pytest~\cite{KPDT2024HTRD}.
Their output is the compared to the text following the snippet in the \gls{docstring}.
If the output matches this text, the test succeeds.
Otherwise it fails.%
}%
}%
%
%
\newglossaryentry{exitCode}{%
name={exit code},%
description={%
When a process terminates, it can return a single integer value~(the exit status code) to indicate success or failure~\cite{J2024PTOGBSI8IS12EETAP}. %
Per convention, an exit code of~0 means success. %
Any non-zero exit code indicates an error. %
Under \python, you can terminate the current process at any time by calling \pythonilIdx{exit} and optionally passing in the exit code that should be returned. %
If \pythonilIdx{exit} is not explicitly called, then the interpreter will return an exit code of~0 once the process normally terminates.%
If the process was terminated by an uncaught \pythonilIdx{Exception}, a non-zero exit code, usually~1, is returned.%
}%
}%
%
%
\newglossaryentry{exponent}{%
name={exponent},%
description={%
The exponent is the part of a floating point number that stores a power of~2 with which the \gls{significand} is multiplied. %
This allows for covering a wide range of different precisions and representing both very large and very small numbers. %
In the 64~bit double precision IEEE~Standard 754 floating point number layout~\cite{IEEE2019ISFFPA,H1997IS7FPN}, the exponent is 11~bits with a \gls{bias} of~1023.%
\optionalRef{sec:howFloatingPointNumbersWork}{ See \cref{sec:howFloatingPointNumbersWork}.}%
}%
}%
%
%
\newglossaryentry{fstring}{%
name={f\nobreakdashes-string},%
plural={f\nobreakdashes-strings},%
sort={f-string},
description={%
is a special string in \python, which delimited by \pythonil{f"..."}\pythonIdx{f\textquotedbl\idxdots\textquotedbl} which can contain expressions in curly braces like \pythonil{f"a\{6-1\}b"} that are then turned to text via \gls{strinterpolation}, which turns the string to~\pythonil{"a5b"}.%
\optionalRef{sec:fstrings}{ f\nobreakdashes-strings are discussed in \cref{sec:fstrings}.}%
}%
}%
%
%
\newglossaryentry{git}{%
name={Git},%
description={%
is a distributed \acrfull{VCS} which allows multiple users to work on the same code while preserving the history of the code changes~\cite{S2023LG,T2024BGAGVCPMATFTND}. %
Learn more at \url{https://git-scm.com}.%
}%
}%
\protected\gdef\git{\pgls{git}}%
%
%
\newglossaryentry{gitee}{%
name={gitee},%
description={%
is a China-based \github\ alternative.
Learn more at \url{https://gitee.com}.%
}%
}%
%
%
\newglossaryentry{github}{%
name={GitHub},%
description={%
is a website where software projects can be hosted and managed via the \gls{git} \gls{VCS}~\cite{PRGWSUdVLFTEKPKFBV2016TSRFTAOGAG,T2024BGAGVCPMATFTND}. %
Learn more at \url{https://github.com}.%
}%
}%
\protected\gdef\github{\pgls{github}}%
%
%
\newglossaryentry{Java}{%
text={\softwareStyle{Java}},%
name={Java},%
sort={Java},%
description={%
is another very successful programming language, with roots in the \gls{C}~family of languages~\cite{LNL2020LJ,B2008EJ}.%
}%
}%
%
%
\newglossaryentry{javascript}{%
name={JavaScript},%
sort={JavaScript},%
description={%
JavaScript is the predominant programming language used in websites to develop interactive contents for display in browsers~\cite{E1999ELS}.%
}%
}%
%
%
\newglossaryentry{locale}{%
name={locale},%
description={%
A \gls{locale} corresponds basically to a selection of country or culture for a system or application~\cite{J2024L}. %
The \gls{locale} then determines a set of country- or culture-dependent settings, such as the text representation for money or dates, or what decimal separators are used.%
A good example is, for example, that American English readers will interpret the date~7/4/2000 as July~4th, 2000, whereas British English readers will read this as the 7th~of April, 2000~\cite{CDE2025D}. %
It is therefore important that software printing dates knows whether it is running on an American or British English~PC.%
}%
}%
%
%
\newglossaryentry{numerator}{%
name={numerator},%
description={%
The number~$a$ of a fraction~$\frac{a}{b}\in\rationalNumbers$ is called the \emph{numerator}.}%
}%
%
%
\newglossaryentry{strinterpolation}{%
name={(string) interpolation},%
sort={string interpolation},
description={%
In \python, string interpolation is the process where all the expressions in an \gls{fstring} are evaluated and the final string is constructed. %
An example for string interpolation is turning \pythonil{f"Rounded \{1.234:.2f\}"} to \pythonil{"Rounded 1.23"}\pythonIdx{.2f}.%
\optionalRef{sec:fstrings}{ This is discussed in \cref{sec:fstrings}.}%
}%
}%
%
%
\newglossaryentry{linter}{%
text={linter},%
name={linter},%
sort={linter},%
description={%
A linter is a tool for analyzing static program code to identify bugs, problems, vulnerabilities, and inconsistent code styles~\cite{J1978LACPC,RJYKK2022CULTDVM}. %
\ruff\ is an example for a linter used in the \python\ world.%
}%
}%
%
%
\newglossaryentry{localhost}{%
name={localhost},%
description={%
is the hostname of the current computer~\cite{RFC2606,RFC6761}. %
It is equivalent to the IP~address~\textil{127.0.0.1}. %
Any message or package sent to localhost will be sent to the current computer itself.%
}%
}%
\protected\gdef\localhost{\pgls{localhost}}%
%
%
\newglossaryentry{mantissa}{%
name={mantissa},%
description={%
See \gls{significand}.%
}%
}%
%
%
\newglossaryentry{modulodiv}{%
name={modulo division},%
description= that computes the remainder of a division. %
\pythonil{15 \% 6} gives us \pythonil{3}.%
\optionalRef{sec:int}{ Modulo division is mentioned in \cref{sec:int}.}%
}%
}%
%
%
\newglossaryentry{package}{%
name={package},%
description={%
A \python\ package is basically a directory containing \python\ files. %
This allows us to group functionality together as a library that can be used by different applications. %
Many popular \python\ packages are offered as open source at \gls{pypi} and can be installed with~\pip.%
\optionalRef{sec:pipAndVenv}{ We discuss this in \cref{sec:pipAndVenv}.}%
}%
}%
%
%
\newglossaryentry{port}{%
name={port},%
description={%
A \gls{port} in networking is a software-defined number associated to a network protocol that receives or transmits communication for a specific service~\cite{KR2020CNATDA}. %
In the \gls{clientServerArchitecture}, the \gls{server} listens for incoming communication connections at a specific \gls{port}. %
The \glspl{client} connect to the network address and \gls{port} number of the \gls{server} to establish such connections. %
\gls{port} numbers range from~0 to~65535, where the \gls{port} numbers from~0 to~1023 are so-called well-known ports corresponding to the most common services, e.g., \gls{HTTP}, the protocol underpinning of the \gls{WWW}, uses normally port~80.%
}%
}%
%
%
\newglossaryentry{rdb}{%
name={relational database},%
description={%
A relational \gls{db} is a database that organizes data into rows~(tuples, records) and columns~(attributes), which collectively form tables~(relations) where the data points are related to each other~\cite{I2021WIARDB,C1970ARMODFLSDB,SC1975OTPOARADI,T2018ISARD,H2016RDDAI,HM2024IMARD,databases}.%
}%
}%
%
%
\newglossaryentry{server}{%
text={server},%
name={server},%
sort={server},%
description={%
In a \gls{clientServerArchitecture}, the \gls{server} is a process that fulfills the requests of the \glspl{client}. %
It usually waits for incoming communication carring the requests from the \glspl{client}. %
For each request, it takes the necessary actions, performs the required computations, and then sends a response with the result of the request. %
Typical examples for \glspl{server} are web servers~\cite{C2022HAFTLS} in the internet as well as \glspl{dbms}. %
It is also common to refer to the computer running the \gls{server} processes as \gls{server} as well, i.e., to call it the \inQuotes{\gls{server} computer}~\cite{L2022MUS}.%
}%
}%
\protected\gdef\server{\pgls{server}}%
%
%
\newglossaryentry{significand}{%
name={significand},%
description={%
The significand is the part of a floating point number that stores the digits of the number~(in binary representation). %
In the 64~bit double precision IEEE~Standard 754 floating point number layout~\cite{IEEE2019ISFFPA,H1997IS7FPN}, the exponent is 52~bits.%
\optionalRef{sec:howFloatingPointNumbersWork}{ See \cref{sec:howFloatingPointNumbersWork}.}%
}%
}%
%
%
\newglossaryentry{signature}{%
name={signature},%
description={%
\pythonIdx{function!signature}%
The signature of a function refers to the parameters and their types, the return type, and the exceptions that the function can raise~\cite{M2023SF}. %
In \python, the function~\pythonilIdx{signature} of the module~\pythonilIdx{inspect} provides some information about the signature of a function~\cite{PEP362}.%
}%
}%
%
%
\newglossaryentry{signBit}{%
name={sign bit},%
description={%
The sign bit indicates whether a floating point number is positive or negative in the 64~bit double precision IEEE~Standard 754 floating point number layout~\cite{IEEE2019ISFFPA,H1997IS7FPN}.%
\optionalRef{sec:howFloatingPointNumbersWork}{ See \cref{sec:howFloatingPointNumbersWork}.}%
}%
}%
%
%
\newglossaryentry{stackTrace}{%
name={stack trace},%
description={%
A stack trace gives information the way in which one function invoked another. %
The term comes from the fact that the data needed to implement function calls is stored in a stack data structure~\cite{K1997FA}. %
The data for the most recently invoked function is on top, the data of the function that called is right below, the data of the function that called that one comes next, and so on. %
Printing a stack trace can be very helpful when trying to find out where an \pythonilIdx{Exception} occurred.%
\optionalRef{sec:exceptions}{ See, for instance, \cref{sec:exceptions}.}%
}%
}%
%
\newglossaryentry{sudo}{%
text={\texttt{sudo}},%
name={sudo},%
sort={sudo},%
description={%
In order to perform administrative tasks such as installing new software under \linux, root~(or ``super'') user privileges as needed~\cite{CN2020ULB}. %
A normal user can execute a program in the \gls{terminal} as super user by pre-pending \bashil{sudo}, often referred to as ``super user do.'' %
This requires the root password.%
}%
}%
%
%
\newglossaryentry{terminal}{%
name={terminal},%
plural={terminals},%
description={%
A terminal is a text-based window where you can enter commands and execute them~\cite{CN2020ULB,B2022ELATCL}. %
Knowing what a terminal is and how to use it is very essential in any programming- or system administration-related task. %
If you want to open a terminal under \microsoftWindows, you can \windowsTerminal. %
\optionalRef{fig:installingPythonWindows01openTerminal}{This is shown, e.g., in \cref{fig:installingPythonWindows01openTerminal}. }%
Under \ubuntu\ \linux, \ubuntuTerminal\ opens a terminal, which then runs a \bash\ shell inside.%
}%
}%
%
%
\newglossaryentry{typeHint}{%
name={type hint},%
plural={type hints},%
description={%
are annotations that help programmers and static code analysis tools such as \mypy\ to better understand what type a variable or function parameter is supposed to be~\cite{PEP484,PEP482}. %
\python\ is a dynamically typed programming language where you do not need to specify the type of, e.g., a variable. %
This creates problems for code analysis, both automated as well as manual: %
For example, it may not always be clear whether a variable or function parameter should be an integer or floating point number. %
The annotations allow us to explicitly state which type is expected. %
They are \emph{ignored} during the program execution. %
They are a basically a piece of documentation.%
\optionalRef{sec:varTypeHints}{ See \cref{sec:varTypeHints}.}%
}%
}%
%
%
\newglossaryentry{unicode}{%
name={Unicode},%
description={%
A standard for assigning characters to numbers~\cite{TUC2023U1510,TUC2023U151ACS,ISOIEC106462020ITUCCSU}. %
The Unicode standard supports basically all characters from all languages that are currently in use, as well as many special symbols. %
It is the predominantly used way to represent characters in computers and is regularly updated and improved.%
}%
}%
%
%
\newglossaryentry{unitTest}{%
name={unit test},%
description={%
Software development is centered around creating the program code of an application, library, or otherwise useful system. %
A \emph{unit test} is an \emph{additional} code fragment that is not part of that productive code. %
It exists to execute~(a part of) the productive code in a certain scenario~(e.g., with specific parameters), to observe the behavior of that code, and to compare whether this behavior meets the specification~\cite{P2021PUTAAOAEUTIP,R2006ASOUTP,TLG2006UTCU}. %
If not, the unit test fails. %
The use of unit tests is at least threefold: %
First, they help us to detect errors in the code. %
Second, program code is usually not developed only once and, from then on, used without change indefinitely. %
Instead, programs are often updated, improved, extended, and maintained over a long time. %
Unit tests can help us to detect whether such changes in the program code, maybe after years, violate the specification or, maybe, cause another, depending, module of the program to violate its specification.
Third, they are part of the documentation or even specification of a program.%
\optionalRef{def:unitTesting}{ See also \cref{def:unitTesting}.}%
}%
}%
%
%
\newglossaryentry{virtualEnvironment}{%
name={virtual environment},%
description={%
A virtual environment is a directory that contains a local \python\ installation~\cite{PSF2024VEAP,PEP405}. %
It comes with its own package installation directory. %
Multiple different virtual environments can be installed on a system. %
This allows different applications to use different versions of the same packages without conflict, because we can simply install these applications into different \glspl{virtualEnvironment}.%
\optionalRef{sec:venvAndPycharm}{ See \cref{sec:venvAndPycharm} for details.}%
}%
}%
%
%
\newglossaryentry{xAxis}{%
name={\ensuremath{x}\nobreakdashes-axis},%
plural={\ensuremath{x}\nobreakdashes-axes},%
sort={x-axis},
description={%
The \ensuremath{x}\nobreakdashes-axis is the horizontal axis of a two-dimensional coordinate system, often referred to abscissa.%
}%
}%
%
%

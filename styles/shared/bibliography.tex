%%
%% Citations and bibliography.
%%
%
%% Load the biblatex package.
\RequirePackage[%
maxbibnames=99,%
maxnames=99,%
maxitems=99,%
backend=biber,%
style=numeric-comp,%
backref=true,%
urldate=iso,%
date=comp,%
]{biblatex}%
%
%% Set our bibliography.
\bibliography{\bookbaseDir/bibliography/bibliography}
%
%% Define formats for ISBNs and ISSNs that link to search engines.
\xdef\isbnSearchEngineA{https://isbnsearch.org/isbn}%
\xdef\isbnSearchEngineB{https://www.directtextbook.com/isbn}%
\protected\gdef\isbn#1{ISBN:~\expandafter\href{\isbnSearchEngineA/#1}{#1}}%
\DeclareFieldFormat{isbn}{\isbn{#1}}%
\protected\gdef\issn#1{ISSN:~\href{https://portal.issn.org/api/search?search[]=MUST=allissnbis=#1}{#1}}%
\DeclareFieldFormat{issn}{\issn{#1}}%
%
%% Print the ISSNs and ISBNs for books.
\RequirePackage{xpatch}%
\@for\next:=book,inbook,collection,incollection,proceedings,inproceedings\do{%
\xpatchbibdriver{\next}%
{{\printfield{isbn}}}%
{{\printfield{issn}\newunit\newblock\printfield{isbn}}}%
{}{}%
}%
%
%% Print the number field as volume for book series.
\DeclareFieldFormat[book,inbook,collection,incollection,proceedings,inproceedings]%
{number}{\bibstring{volume}~#1 \bibstring{ofseries}}%
\renewbibmacro*{series+number}{%
\printfield{number}%
\setunit*{\addspace}%
\printfield{series}%
\newunit%
}%
%
%% Allow linebreaks in URLs and dois.
\setcounter{biburlnumpenalty}{8000}%
\setcounter{biburlucpenalty}{8000}%
\setcounter{biburllcpenalty}{8000}%
%
%% Make sure that the prefixes of dois and URLs have the right formats.
\DeclareFieldFormat{doi}{%
doi\addcolon%
\ifhyperref%
{\href{https://doi.org/#1}{\nolinkurl{#1}}}%
{\nolinkurl{#1}}}%
%
\DeclareFieldFormat{url}{%
URL\addcolon\space%
\ifhyperref%
{\href{#1}{\nolinkurl{#1}}}%
{\nolinkurl{#1}}}%
%
%% Get the volume(number):pages format for articles.
\renewbibmacro*{volume+number+eid}{%
\printfield{volume}%
\printfield[parens]{number}%
\setunit{\addcolon}%
\unspace\printfield{pages}}%
\DeclareFieldFormat[article]{number}{\mkbibparens{#1}}
\renewbibmacro*{issue+date}{}%
\renewbibmacro*{note+pages}{%
\setunit{\addcomma\addspace}%
\usebibmacro{date}\newunit\newblock\printfield{note}}%
%
\DeclareFieldFormat[article]{pages}{#1}
%
%% Remove "In:" from articles.
\renewbibmacro{in:}{%
\ifentrytype{article}{}{\bibstring{in}%
\printunit{\intitlepunct}}}%
%
\newrobustcmd*{\citeauthorfull}{\AtNextCite{\DeclareNameAlias{labelname}{given-family}}\citeauthor}%
%
%
%%
%% A command for quotations with citation.
%% #1 the citation reference that is quotes
%% #2 the quotation
\protected\long\gdef\cquotation#1#2{%
\begin{quotation}{\rmfamily{%
#2\strut\\%
\strut\hfill\strut---~{\small{\textit{\citeauthorfull{#1}~\cite{#1}, \citeyear{#1}}}}\strut%
}}\end{quotation}%
}%
%
%%% Load the slides bibliography commands, if need be
\ifIsSlides\expandafter\@firstoftwo\else\expandafter\@secondoftwo\fi{%
%%
%% Citations and bibliography.
%%
%
%% Load the biblatex package.
\RequirePackage[%
maxbibnames=99,%
maxnames=99,%
maxitems=99,%
backend=biber,%
style=numeric-comp,%
backref=true,%
urldate=iso,%
]{biblatex}%
%
%% Set our bibliography.
\bibliography{\bookbaseDir/bibliography/bibliography}
%
%% Define formats for ISBNs and ISSNs that link to search engines.
\xdef\isbnSearchEngineA{https://isbnsearch.org/isbn}%
\xdef\isbnSearchEngineB{https://www.directtextbook.com/isbn}%
\protected\gdef\isbn#1{ISBN:~\expandafter\href{\isbnSearchEngineA/#1}{#1}}%
\DeclareFieldFormat{isbn}{\isbn{#1}}%
\protected\gdef\issn#1{ISSN:~\href{https://portal.issn.org/api/search?search[]=MUST=allissnbis=#1}{#1}}%
\DeclareFieldFormat{issn}{\issn{#1}}%
%
%% Print the ISSNs and ISBNs for books.
\RequirePackage{xpatch}%
\@for\next:=book,inbook,collection,incollection,proceedings,inproceedings\do{%
\xpatchbibdriver{\next}%
{{\printfield{isbn}}}%
{{\printfield{issn}\newunit\newblock\printfield{isbn}}}%
{}{}%
}%
%
%% Print the number field as volume for book series.
\DeclareFieldFormat[book,inbook,collection,incollection,proceedings,inproceedings]%
{number}{\bibstring{volume}~#1 \bibstring{ofseries}}%
\renewbibmacro*{series+number}{%
\printfield{number}%
\setunit*{\addspace}%
\printfield{series}%
\newunit%
}%
%
%% Allow linebreaks in URLs and dois.
\setcounter{biburlnumpenalty}{8000}%
\setcounter{biburlucpenalty}{8000}%
\setcounter{biburllcpenalty}{8000}%
%
%% Make sure that the prefixes of dois and URLs have the right formats.
\DeclareFieldFormat{doi}{%
doi\addcolon%
\ifhyperref%
{\href{https://doi.org/#1}{\nolinkurl{#1}}}%
{\nolinkurl{#1}}}%
%
\DeclareFieldFormat{url}{%
URL\addcolon\space%
\ifhyperref%
{\href{#1}{\nolinkurl{#1}}}%
{\nolinkurl{#1}}}%
%
%% Get the volume(number):pages format for articles.
\renewbibmacro*{volume+number+eid}{%
\printfield{volume}%
\printfield[parens]{number}%
\setunit{\addcolon}%
\unspace\printfield{pages}}%
\DeclareFieldFormat[article]{number}{\mkbibparens{#1}}
\renewbibmacro*{issue+date}{}%
\renewbibmacro*{note+pages}{%
\setunit{\addcomma\addspace}%
\usebibmacro{date}\newunit\newblock\printfield{note}}%
%
\DeclareFieldFormat[article]{pages}{#1}
%
%% Remove "In:" from articles.
\renewbibmacro{in:}{%
\ifentrytype{article}{}{\bibstring{in}%
\printunit{\intitlepunct}}}%
%
\newrobustcmd*{\citeauthorfull}{\AtNextCite{\DeclareNameAlias{labelname}{given-family}}\citeauthor}%
%
%
%%
%% A command for quotations with citation.
%% #1 the citation reference that is quotes
%% #2 the quotation
\protected\long\gdef\cquotation#1#2{%
\begin{quotation}{\rmfamily{%
#2\\%
\strut\hfill\strut---~{\small{\textit{\citeauthorfull{#1}~\cite{#1}, \citeyear{#1}}}}\strut%
}}\end{quotation}%
}%
%
%%% Load the slides bibliography commands, if need be
\ifIsSlides\expandafter\@firstoftwo\else\expandafter\@secondoftwo\fi{%
%%
%% Citations and bibliography.
%%
%
%% Load the biblatex package.
\RequirePackage[%
maxbibnames=99,%
maxnames=99,%
maxitems=99,%
backend=biber,%
style=numeric-comp,%
backref=true,%
urldate=iso,%
]{biblatex}%
%
%% Set our bibliography.
\bibliography{\bookbaseDir/bibliography/bibliography}
%
%% Define formats for ISBNs and ISSNs that link to search engines.
\xdef\isbnSearchEngineA{https://isbnsearch.org/isbn}%
\xdef\isbnSearchEngineB{https://www.directtextbook.com/isbn}%
\protected\gdef\isbn#1{ISBN:~\expandafter\href{\isbnSearchEngineA/#1}{#1}}%
\DeclareFieldFormat{isbn}{\isbn{#1}}%
\protected\gdef\issn#1{ISSN:~\href{https://portal.issn.org/api/search?search[]=MUST=allissnbis=#1}{#1}}%
\DeclareFieldFormat{issn}{\issn{#1}}%
%
%% Print the ISSNs and ISBNs for books.
\RequirePackage{xpatch}%
\@for\next:=book,inbook,collection,incollection,proceedings,inproceedings\do{%
\xpatchbibdriver{\next}%
{{\printfield{isbn}}}%
{{\printfield{issn}\newunit\newblock\printfield{isbn}}}%
{}{}%
}%
%
%% Print the number field as volume for book series.
\DeclareFieldFormat[book,inbook,collection,incollection,proceedings,inproceedings]%
{number}{\bibstring{volume}~#1 \bibstring{ofseries}}%
\renewbibmacro*{series+number}{%
\printfield{number}%
\setunit*{\addspace}%
\printfield{series}%
\newunit%
}%
%
%% Allow linebreaks in URLs and dois.
\setcounter{biburlnumpenalty}{8000}%
\setcounter{biburlucpenalty}{8000}%
\setcounter{biburllcpenalty}{8000}%
%
%% Make sure that the prefixes of dois and URLs have the right formats.
\DeclareFieldFormat{doi}{%
doi\addcolon%
\ifhyperref%
{\href{https://doi.org/#1}{\nolinkurl{#1}}}%
{\nolinkurl{#1}}}%
%
\DeclareFieldFormat{url}{%
URL\addcolon\space%
\ifhyperref%
{\href{#1}{\nolinkurl{#1}}}%
{\nolinkurl{#1}}}%
%
%% Get the volume(number):pages format for articles.
\renewbibmacro*{volume+number+eid}{%
\printfield{volume}%
\printfield[parens]{number}%
\setunit{\addcolon}%
\unspace\printfield{pages}}%
\DeclareFieldFormat[article]{number}{\mkbibparens{#1}}
\renewbibmacro*{issue+date}{}%
\renewbibmacro*{note+pages}{%
\setunit{\addcomma\addspace}%
\usebibmacro{date}\newunit\newblock\printfield{note}}%
%
\DeclareFieldFormat[article]{pages}{#1}
%
%% Remove "In:" from articles.
\renewbibmacro{in:}{%
\ifentrytype{article}{}{\bibstring{in}%
\printunit{\intitlepunct}}}%
%
\newrobustcmd*{\citeauthorfull}{\AtNextCite{\DeclareNameAlias{labelname}{given-family}}\citeauthor}%
%
%
%%
%% A command for quotations with citation.
%% #1 the citation reference that is quotes
%% #2 the quotation
\protected\long\gdef\cquotation#1#2{%
\begin{quotation}{\rmfamily{%
#2\\%
\strut\hfill\strut---~{\small{\textit{\citeauthorfull{#1}~\cite{#1}, \citeyear{#1}}}}\strut%
}}\end{quotation}%
}%
%
%%% Load the slides bibliography commands, if need be
\ifIsSlides\expandafter\@firstoftwo\else\expandafter\@secondoftwo\fi{%
%%
%% Citations and bibliography.
%%
%
%% Load the biblatex package.
\RequirePackage[%
maxbibnames=99,%
maxnames=99,%
maxitems=99,%
backend=biber,%
style=numeric-comp,%
backref=true,%
urldate=iso,%
]{biblatex}%
%
%% Set our bibliography.
\bibliography{\bookbaseDir/bibliography/bibliography}
%
%% Define formats for ISBNs and ISSNs that link to search engines.
\xdef\isbnSearchEngineA{https://isbnsearch.org/isbn}%
\xdef\isbnSearchEngineB{https://www.directtextbook.com/isbn}%
\protected\gdef\isbn#1{ISBN:~\expandafter\href{\isbnSearchEngineA/#1}{#1}}%
\DeclareFieldFormat{isbn}{\isbn{#1}}%
\protected\gdef\issn#1{ISSN:~\href{https://portal.issn.org/api/search?search[]=MUST=allissnbis=#1}{#1}}%
\DeclareFieldFormat{issn}{\issn{#1}}%
%
%% Print the ISSNs and ISBNs for books.
\RequirePackage{xpatch}%
\@for\next:=book,inbook,collection,incollection,proceedings,inproceedings\do{%
\xpatchbibdriver{\next}%
{{\printfield{isbn}}}%
{{\printfield{issn}\newunit\newblock\printfield{isbn}}}%
{}{}%
}%
%
%% Print the number field as volume for book series.
\DeclareFieldFormat[book,inbook,collection,incollection,proceedings,inproceedings]%
{number}{\bibstring{volume}~#1 \bibstring{ofseries}}%
\renewbibmacro*{series+number}{%
\printfield{number}%
\setunit*{\addspace}%
\printfield{series}%
\newunit%
}%
%
%% Allow linebreaks in URLs and dois.
\setcounter{biburlnumpenalty}{8000}%
\setcounter{biburlucpenalty}{8000}%
\setcounter{biburllcpenalty}{8000}%
%
%% Make sure that the prefixes of dois and URLs have the right formats.
\DeclareFieldFormat{doi}{%
doi\addcolon%
\ifhyperref%
{\href{https://doi.org/#1}{\nolinkurl{#1}}}%
{\nolinkurl{#1}}}%
%
\DeclareFieldFormat{url}{%
URL\addcolon\space%
\ifhyperref%
{\href{#1}{\nolinkurl{#1}}}%
{\nolinkurl{#1}}}%
%
%% Get the volume(number):pages format for articles.
\renewbibmacro*{volume+number+eid}{%
\printfield{volume}%
\printfield[parens]{number}%
\setunit{\addcolon}%
\unspace\printfield{pages}}%
\DeclareFieldFormat[article]{number}{\mkbibparens{#1}}
\renewbibmacro*{issue+date}{}%
\renewbibmacro*{note+pages}{%
\setunit{\addcomma\addspace}%
\usebibmacro{date}\newunit\newblock\printfield{note}}%
%
\DeclareFieldFormat[article]{pages}{#1}
%
%% Remove "In:" from articles.
\renewbibmacro{in:}{%
\ifentrytype{article}{}{\bibstring{in}%
\printunit{\intitlepunct}}}%
%
\newrobustcmd*{\citeauthorfull}{\AtNextCite{\DeclareNameAlias{labelname}{given-family}}\citeauthor}%
%
%
%%
%% A command for quotations with citation.
%% #1 the citation reference that is quotes
%% #2 the quotation
\protected\long\gdef\cquotation#1#2{%
\begin{quotation}{\rmfamily{%
#2\\%
\strut\hfill\strut---~{\small{\textit{\citeauthorfull{#1}~\cite{#1}, \citeyear{#1}}}}\strut%
}}\end{quotation}%
}%
%
%%% Load the slides bibliography commands, if need be
\ifIsSlides\expandafter\@firstoftwo\else\expandafter\@secondoftwo\fi{%
\input{\bookbaseDir/styles/slides/bibliography.tex}%
}{}%
%
%%% Also print publisher and publisher's address for journal articles
%% Warning: This overwrites any `note` field.
% We transform all article records and put the "address:~publisher"
% in front of the original note, if any.
% This means that we can only use note fields with articles.
\DeclareSourcemap{%
\maps[datatype=bibtex]{%
\map[overwrite]{%
%
%% Resolve publisher in xdata records and pre-pend to note
% We only work on xdata records that start with "j_", because
% in our bibliography, these and only these identify journals.
\pertype{xdata}%
\step[fieldset=entrykey, match=\regexp{\Aj_[a-z].*}, final]%
\step[fieldsource=note, fieldtarget=___tmp___bib]%
\step[fieldset=note, null]%
\step[fieldsource=publisher, final]%
\step[fieldset=address, fieldvalue={:~}, appendstrict]%
\step[fieldset=address, origfieldval, append]%
\step[fieldsource=address, final]%
\step[fieldset=note, origfieldval]%
\step[fieldsource=___tmp___bib, final]%
\step[fieldset=note, fieldvalue={. }, appendstrict]%
\step[fieldset=note, origfieldval, append]%
\step[fieldset=___tmp___bib, null]%
\step[fieldset=publisher, null]%
\step[fieldset=address, null]%
}%
%
%% Resolve publisher in article records and pre-pend to note
\map[overwrite]{%
\pertype{article}%
\step[fieldsource=note, fieldtarget=___tmp___bib]%
\step[fieldset=note, null]%
\step[fieldsource=publisher, final]%
\step[fieldset=address, fieldvalue={:~}, appendstrict]%
\step[fieldset=address, origfieldval, append]%
\step[fieldsource=address, final]%
\step[fieldset=note, origfieldval]%
\step[fieldsource=___tmp___bib, final]%
\step[fieldset=note, fieldvalue={. }, appendstrict]%
\step[fieldset=note, origfieldval, append]%
\step[fieldset=___tmp___bib, null]%
\step[fieldset=publisher, null]%
\step[fieldset=address, null]%
}%
}%
}%
%
%% Fix event date format to eventdate, venue.
%%
\renewbibmacro*{event+venue+date}{%
\printfield{eventtitle}%
\newunit%
\printfield{eventtitleaddon}%%
\newunit%
\printeventdate%
\setunit*{\addcomma\space}%
\printfield{venue}%
\newunit}%
%
%% A link to a github user name
\protected\gdef\githubUser#1{\inQuotes{\href{https://github.com/#1}{#1}}}%
%
%
}{}%
%
%%% Also print publisher and publisher's address for journal articles
%% Warning: This overwrites any `note` field.
% We transform all article records and put the "address:~publisher"
% in front of the original note, if any.
% This means that we can only use note fields with articles.
\DeclareSourcemap{%
\maps[datatype=bibtex]{%
\map[overwrite]{%
%
%% Resolve publisher in xdata records and pre-pend to note
% We only work on xdata records that start with "j_", because
% in our bibliography, these and only these identify journals.
\pertype{xdata}%
\step[fieldset=entrykey, match=\regexp{\Aj_[a-z].*}, final]%
\step[fieldsource=note, fieldtarget=___tmp___bib]%
\step[fieldset=note, null]%
\step[fieldsource=publisher, final]%
\step[fieldset=address, fieldvalue={:~}, appendstrict]%
\step[fieldset=address, origfieldval, append]%
\step[fieldsource=address, final]%
\step[fieldset=note, origfieldval]%
\step[fieldsource=___tmp___bib, final]%
\step[fieldset=note, fieldvalue={. }, appendstrict]%
\step[fieldset=note, origfieldval, append]%
\step[fieldset=___tmp___bib, null]%
\step[fieldset=publisher, null]%
\step[fieldset=address, null]%
}%
%
%% Resolve publisher in article records and pre-pend to note
\map[overwrite]{%
\pertype{article}%
\step[fieldsource=note, fieldtarget=___tmp___bib]%
\step[fieldset=note, null]%
\step[fieldsource=publisher, final]%
\step[fieldset=address, fieldvalue={:~}, appendstrict]%
\step[fieldset=address, origfieldval, append]%
\step[fieldsource=address, final]%
\step[fieldset=note, origfieldval]%
\step[fieldsource=___tmp___bib, final]%
\step[fieldset=note, fieldvalue={. }, appendstrict]%
\step[fieldset=note, origfieldval, append]%
\step[fieldset=___tmp___bib, null]%
\step[fieldset=publisher, null]%
\step[fieldset=address, null]%
}%
}%
}%
%
%% Fix event date format to eventdate, venue.
%%
\renewbibmacro*{event+venue+date}{%
\printfield{eventtitle}%
\newunit%
\printfield{eventtitleaddon}%%
\newunit%
\printeventdate%
\setunit*{\addcomma\space}%
\printfield{venue}%
\newunit}%
%
%% A link to a github user name
\protected\gdef\githubUser#1{\inQuotes{\href{https://github.com/#1}{#1}}}%
%
%
}{}%
%
%%% Also print publisher and publisher's address for journal articles
%% Warning: This overwrites any `note` field.
% We transform all article records and put the "address:~publisher"
% in front of the original note, if any.
% This means that we can only use note fields with articles.
\DeclareSourcemap{%
\maps[datatype=bibtex]{%
\map[overwrite]{%
%
%% Resolve publisher in xdata records and pre-pend to note
% We only work on xdata records that start with "j_", because
% in our bibliography, these and only these identify journals.
\pertype{xdata}%
\step[fieldset=entrykey, match=\regexp{\Aj_[a-z].*}, final]%
\step[fieldsource=note, fieldtarget=___tmp___bib]%
\step[fieldset=note, null]%
\step[fieldsource=publisher, final]%
\step[fieldset=address, fieldvalue={:~}, appendstrict]%
\step[fieldset=address, origfieldval, append]%
\step[fieldsource=address, final]%
\step[fieldset=note, origfieldval]%
\step[fieldsource=___tmp___bib, final]%
\step[fieldset=note, fieldvalue={. }, appendstrict]%
\step[fieldset=note, origfieldval, append]%
\step[fieldset=___tmp___bib, null]%
\step[fieldset=publisher, null]%
\step[fieldset=address, null]%
}%
%
%% Resolve publisher in article records and pre-pend to note
\map[overwrite]{%
\pertype{article}%
\step[fieldsource=note, fieldtarget=___tmp___bib]%
\step[fieldset=note, null]%
\step[fieldsource=publisher, final]%
\step[fieldset=address, fieldvalue={:~}, appendstrict]%
\step[fieldset=address, origfieldval, append]%
\step[fieldsource=address, final]%
\step[fieldset=note, origfieldval]%
\step[fieldsource=___tmp___bib, final]%
\step[fieldset=note, fieldvalue={. }, appendstrict]%
\step[fieldset=note, origfieldval, append]%
\step[fieldset=___tmp___bib, null]%
\step[fieldset=publisher, null]%
\step[fieldset=address, null]%
}%
}%
}%
%
%% Fix event date format to eventdate, venue.
%%
\renewbibmacro*{event+venue+date}{%
\printfield{eventtitle}%
\newunit%
\printfield{eventtitleaddon}%%
\newunit%
\printeventdate%
\setunit*{\addcomma\space}%
\printfield{venue}%
\newunit}%
%
%% A link to a github user name
\protected\gdef\githubUser#1{\inQuotes{\href{https://github.com/#1}{#1}}}%
%
%
}{}%
%
%%% Also print publisher and publisher's address for journal articles
%% Warning: This overwrites any `note` field.
% We transform all article records and put the "address:~publisher"
% in front of the original note, if any.
% This means that we can only use note fields with articles.
\DeclareSourcemap{%
\maps[datatype=bibtex]{%
\map[overwrite]{%
%
%% Resolve publisher in xdata records and pre-pend to note
% We only work on xdata records that start with "j_", because
% in our bibliography, these and only these identify journals.
\pertype{xdata}%
\step[fieldset=entrykey, match=\regexp{\Aj_[a-z].*}, final]%
\step[fieldsource=entrykey, match=\regexp{\Aj_[a-z].*}, final]%
\step[fieldsource=note, fieldtarget=___tmp___bib]%
\step[fieldset=note, null]%
\step[fieldsource=publisher, final]%
\step[fieldset=address, fieldvalue={:~}, appendstrict]%
\step[fieldset=address, origfieldval, append]%
\step[fieldsource=address, final]%
\step[fieldset=note, origfieldval]%
\step[fieldsource=___tmp___bib, final]%
\step[fieldset=note, fieldvalue={. }, appendstrict]%
\step[fieldset=note, origfieldval, append]%
\step[fieldset=___tmp___bib, null]%
\step[fieldset=publisher, null]%
\step[fieldset=address, null]%
}%
%
%% Resolve publisher in article records and pre-pend to note
\map[overwrite]{%
\pertype{article}%
\step[fieldsource=note, fieldtarget=___tmp___bib]%
\step[fieldset=note, null]%
\step[fieldsource=publisher, final]%
\step[fieldset=address, fieldvalue={:~}, appendstrict]%
\step[fieldset=address, origfieldval, append]%
\step[fieldsource=address, final]%
\step[fieldset=note, origfieldval]%
\step[fieldsource=___tmp___bib, final]%
\step[fieldset=note, fieldvalue={. }, appendstrict]%
\step[fieldset=note, origfieldval, append]%
\step[fieldset=___tmp___bib, null]%
\step[fieldset=publisher, null]%
\step[fieldset=address, null]%
}%
}%
}%
%
%% Fix event date format to eventdate, venue.
%%
\renewbibmacro*{event+venue+date}{%
\printfield{eventtitle}%
\newunit%
\printfield{eventtitleaddon}%%
\newunit%
\printeventdate%
\setunit*{\addcomma\space}%
\printfield{venue}%
\newunit}%
%
%% A link to a github user name
\protected\gdef\githubUser#1{\inQuotes{\href{https://github.com/#1}{#1}}}%
%
%% This command is useful to print bibliography entries in a list for further reading.
%% It works like \cite{...}, but suppresses the url visit date and the publisher address.
\DeclareCiteCommand{\furtherReading}%
{\usebibmacro{prenote}}%
{%
\clearfield{urlyear}\clearname{urlyear}\clearlist{urlyear}%
\clearfield{address}\clearname{address}\clearlist{address}%
\clearfield{location}\clearname{location}\clearlist{location}%
\clearfield{addendum}\clearname{addendum}\clearlist{addendum}%
\clearfield{note}\clearname{note}\clearlist{note}%
\clearfield{notes}\clearname{notes}\clearlist{notes}%
\clearfield{series}\clearname{series}\clearlist{series}%
\clearfield{volume}\clearname{volume}\clearlist{volume}%
\usedriver{%
}{\thefield{entrytype}}%
}%
{\multicitedelim}%
{\usebibmacro{postnote}}%

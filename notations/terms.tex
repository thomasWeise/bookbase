%% Access to Source Codes
%%
%
%
\newglossaryentry{bias}{%
name={bias},%
description={%
The bias is subtracted from the value stored in the \pgls{exponent} field of a floating point number. %
This allows for representing both positive and negative exponents. %
In the 64~bit double precision IEEE~Standard 754 floating point number layout~\cite{IEEE2019ISFFPA,H1997IS7FPN}, the bias is~1023.%
\optionalRef{sec:howFloatingPointNumbersWork}{ See \cref{sec:howFloatingPointNumbersWork}.}%
}%
}%
%
%
\newglossaryentry{breakpoint}{%
name={breakpoint},%
description={%
A breakpoint is a mark in a line of code in an~\pgls{IDE} at which the \debugger\ will pause the execution of a program.%
}%
}%
%
%
\newglossaryentry{C}{%
text={\softwareStyle{C}},%
name={C},%
sort={C},%
description={%
is a programming language, which is very successful in system programming situations~\cite{ISOIEC98892017PLCWDOS,D2024MCFABAFITTCPL}.%
}%
}%
%
%
\newglossaryentry{client}{%
text={client},%
name={client},%
sort={client},%
description={%
In a \pgls{clientServerArchitecture}, the \pgls{client} is a device or process that requests a service from the \pgls{server}. %
It initiates the communication with the \pgls{server}, sends a request, and receives the response with the result of the request. %
Typical examples for \pglspl{client} are web browsers in the internet as well as \pglspl{client} for \pglspl{dbms}, such as \psql.%
}%
}%
%
%
\newglossaryentry{clientServerArchitecture}{%
text={\pgls{client}\nobreakdashes-\pgls{server} architecture},%
name={client-server architecture},%
sort={client-server architecture},%
description={%
is a system design where a central \pgls{server} receives requests from one or multiple \pglspl{client}~\cite{RCKS2019PNP,B1996CSA,OHE1999CSSG,RF2020FOSAAEA,EOEBLGGCH2025CSA}. %
These requests and responses are usually sent over network connections. %
A typical example for such a system is the world wide web~(WWW), where web \pglspl{server} host websites and make them available to web browsers, the \pglspl{client}. %
Another typical example is the structure of \pgls{db} software, where a central \pgls{server}, the \pgls{dbms}, offers access to the \pgls{db} to the different \pglspl{client}. %
Here, the \pgls{client} can be some \pgls{terminal} software shipping with the \pgls{DBMS}, such as \psql, or the different applications that access the \pglspl{db}.%
}%
}%
%
%
\newglossaryentry{csv}{%
name={CSV},%
description={%
A very common and simple text format for exchanging tabular or matrix data is \acrfull{CSV}~\cite{S2005CFAMTFCSVCF}. %
Each row in the text file represents one row in the table or matrix. %
The elements in the row are separated by a fixed delimiter, usually a comma~(\inQuotes{,}), sometimes a semicolon~(\inQuotes{;}). %
\python\ offers some out-of-the-box \pgls{CSV} support in the \pythonilIdx{csv}~module~\cite{PSF2024CCFRAW}.%
}%
}%
%
%
\newglossaryentry{db}{%
name={database},%
description={%
A \acrfull{DB} is an organized collection of structured information or data, typically stored electronically in a computer system. %
Databases are discussed in our book~\cite{databases}.%
}%
}%
%
%
\newglossaryentry{dba}{%
name={database},%
description={%
A \acrfull{DBA} is the person or group responsible for the effective use of database technology in an organization or enterprise.%
}%
}%
%
%
\newglossaryentry{dbms}{%
name={database management system},%
description={%
A \acrfull{DBMS} is the software layer located between the user or application and the~\pgls{db}. %
The \pgls{DBMS} allows the user/application to create, read, write, update, delete, and otherwise manipulate the data in the~\pgls{db}~\cite{YM2024DDMSD}.%
}%
}%
\protected\gdef\db{\pgls{dbms}}%%
%
%
\newglossaryentry{dbs}{%
name={database system},%
description={%
A \acrfull{DBS} is the combination of a \pgls{db} and a the corresponding \pgls{dbms}, i.e., basically, an installation of a \pgls{DBMS} on a computer together with one (or multiple) \pglspl{db}.%
}%
}%
\protected\gdef\db{\pgls{dbms}}%%
%
%
\newglossaryentry{docstring}{%
name={docstring},%
description={%
Docstrings are special string constants in \python\ that contain documentation for modules or functions~\cite{PEP257}. %
They must be delimited by~\pythonil{"""..."""}\pythonIdx{\textquotedbl\textquotedbl\textquotedbl\idxdots\textquotedbl\textquotedbl\textquotedbl}~\cite{PEP257,PEP8}.%
}%
}%
%
%
\newglossaryentry{denominator}{%
name={denominator},%
description={%
The number~$b$ of a fraction~$\frac{a}{b}\in\rationalNumbers$ is called the \emph{denominator}.}%
}%
%
%
\newglossaryentry{doctest}{%
name={doctest},%
description={%
\emph{doctests} are small pieces of code in the \pglspl{docstring} that look like interactive \python\ sessions.
The first line of a statement in such a \python\ snippet is indented with \python{>>>}\pythonIdx{>\strut>\strut>} and the following lines by \pythonil{...}\pythonIdx{\idxdots}.
These snippets can be executed by modules like \pythonilIdx{doctest}~\cite{PSF2024DTIPE} or tools such as \pytest~\cite{KPDT2024HTRD}.
Their output is the compared to the text following the snippet in the \pgls{docstring}.
If the output matches this text, the test succeeds.
Otherwise it fails.%
}%
}%
%
%
\newglossaryentry{exitCode}{%
name={exit code},%
description={%
When a process terminates, it can return a single integer value (the exit status code) to indicate success or failure~\cite{J2024PTOGBSI8IS12EETAP}. %
Per convention, an exit code of~0 means success. %
Any non-zero exit code indicates an error. %
Under \python, you can terminate the current process at any time by calling \pythonilIdx{exit} and optionally passing in the exit code that should be returned. %
If \pythonilIdx{exit} is not explicitly called, then the interpreter will return an exit code of~0 once the process normally terminates.%
If the process was terminated by an uncaught \pythonilIdx{Exception}, a non-zero exit code, usually~1, is returned.%
}%
}%
%
%
\newglossaryentry{exponent}{%
name={exponent},%
description={%
The exponent is the part of a floating point number that stores a power of~2 with which the \pgls{significand} is multiplied. %
This allows for covering a wide range of different precisions and representing both very large and very small numbers. %
In the 64~bit double precision IEEE~Standard 754 floating point number layout~\cite{IEEE2019ISFFPA,H1997IS7FPN}, the exponent is 11~bits with a \pgls{bias} of~1023.%
\optionalRef{sec:howFloatingPointNumbersWork}{ See \cref{sec:howFloatingPointNumbersWork}.}%
}%
}%
%
%
\newglossaryentry{fstring}{%
name={f\nobreakdashes-string},%
plural={f\nobreakdashes-strings},%
sort={f-string},
description={%
is a special string in \python, which delimited by \pythonil{f"..."}\pythonIdx{f\textquotedbl\idxdots\textquotedbl} which can contain expressions in curly braces like \pythonil{f"a\{6-1\}b"} that are then turned to text via \pgls{strinterpolation}, which turns the string to~\pythonil{"a5b"}.%
\optionalRef{sec:fstrings}{ f\nobreakdashes-strings are discussed in \cref{sec:fstrings}.}%
}%
}%
%
%
\newglossaryentry{git}{%
name={Git},%
description={%
Git is a distributed \acrfull{VCS} which allows multiple users to work on the same code while preserving the history of the code changes~\cite{S2023LG,T2024BGAGVCPMATFTND}.%
}%
}%
\protected\gdef\git{\pgls{git}}%
%
%
\newglossaryentry{github}{%
name={GitHub},%
description={%
GitHub is a website where software projects can be hosted and managed via the \pgls{git} \pgls{vcs}~\cite{PRGWSUdVLFTEKPKFBV2016TSRFTAOGAG,T2024BGAGVCPMATFTND}. %
Learn more at \url{https://github.com}.%
}%
}%
\protected\gdef\github{\pgls{github}}%
%
%
\newglossaryentry{goldenRatio}{%
name={golden ratio},%
description={%
The golden ratio or golden section, see the symbol~\numberGoldenRatio.%
}%
}%
%
%
\newglossaryentry{ide}{%
name={Integrated Development Environment},%
sort={Integrated Development Environment},%
description={%
An \acrfull{IDE} is a program that allows the user do multiple different activities required for software development in one single system. %
It often offers functionality such as editing source code, debugging, testing, or interaction with a distributed version control system. %
For this course, we recommend using \pycharm.%
}%
}%
%
%
\newglossaryentry{Java}{%
text={\softwareStyle{Java}},%
name={Java},%
sort={Java},%
description={%
is another very successful programming language, with roots in the \pgls{C}~family of languages~\cite{LNL2020LJ,B2008EJ}.%
}%
}%
%
%
\newglossaryentry{javascript}{%
name={JavaScript},%
sort={JavaScript},%
description={%
JavaScript is the predominant programming language used in websites to develop interactive contents for display in browsers~\cite{E1999ELS}.%
}%
}%
%
%
\newglossaryentry{json}{%
name={JSON},%
sort={JSON},%
description={%
\acrfull{JSON} is a data interchange format~\cite{E2017SE4TJDIS,B2017TJONJDIF} based on \pgls{javascript}~\cite{E1999ELS} syntax.%
}%
}%
%
%
\newglossaryentry{locale}{%
name={locale},%
description={%
A \pgls{locale} corresponds basically to a selection of country or culture for a system or application~\cite{J2024L}. %
The \pgls{locale} then determines a set of country- or culture-dependent settings, such as the text representation for money or dates, or what decimal separators are used.%
A good example is, for example, that American English readers will interpret the date~7/4/2000 as July~4th, 2000, whereas British English readers will read this as the 7th~of April, 2000~\cite{CDE2025D}. %
It is therefore important that software printing dates knows whether it is running on an American or British English~PC.%
}%
}%
%
%
\newglossaryentry{numerator}{%
name={numerator},%
description={%
The number~$a$ of a fraction~$\frac{a}{b}\in\rationalNumbers$ is called the \emph{numerator}.}%
}%
%
%
\newglossaryentry{strinterpolation}{%
name={(string) interpolation},%
sort={string interpolation},
description={%
In \python, string interpolation is the process where all the expressions in an \pgls{fstring} are evaluated and the final string is constructed. %
An example for string interpolation is turning \pythonil{f"Rounded \{1.234:.2f\}"} to \pythonil{"Rounded 1.23"}\pythonIdx{.2f}.%
\optionalRef{sec:fstrings}{ This is discussed in \cref{sec:fstrings}.}%
}%
}%
%
%
\newglossaryentry{linter}{%
text={linter},%
name={linter},%
sort={linter},%
description={%
A linter is a tool for analyzing static program code to identify bugs, problems, vulnerabilities, and inconsistent code styles~\cite{J1978LACPC,RJYKK2022CULTDVM}. %
\ruff\ is an example for a linter used in the \python\ world.%
}%
}%
%
%
\newglossaryentry{mantissa}{%
name={mantissa},%
description={%
See \gls{significand}.%
}%
}%
%
%
\newglossaryentry{modulodiv}{%
name={modulo division},%
description= that computes the remainder of a division. %
\pythonil{15 \% 6} gives us \pythonil{3}.%
\optionalRef{sec:int}{ Modulo division is mentioned in \cref{sec:int}.}%
}%
}%
%
%
\newglossaryentry{package}{%
name={package},%
description={%
A \python\ package is basically a directory containing \python\ files. %
This allows us to group functionality together as a library that can be used by different applications. %
Many popular \python\ packages are offered as open source at \pgls{pypi} and can be installed with~\pip.%
\optionalRef{sec:pipAndVenv}{ We discuss this in \cref{sec:pipAndVenv}.}%
}%
}%
%
%
\newglossaryentry{port}{%
name={port},%
description={%
A \pgls{port} in networking is a software-defined number associated to a network protocol that receives or transmits communication for a specific service~\cite{KR2020CNATDA}. %
In the \pgls{clientServerArchitecture}, the \pgls{server} listens for incoming communication connections at a specific \pgls{port}. %
The \pglspl{client} connect to the network address and \pgls{port} number of the \pgls{server} to establish such connections. %
\Pgls{port} numbers range from~0 to~65535, where the \pgls{port} numbers from~0 to~1023 are so-called well-known ports corresponding to the most common services, e.g., HTTP, the protocol underpinning of the WWW, uses normally port~80.%
}%
}%
%
%
\newglossaryentry{regexp}{%
name={regular expression},%
description={%
Regular expressions, often called \inQuotes{regex} for short, are sequences of characters that define a search pattern for text strings~\cite{K2024REH,PSF2024RREO,N2018RQSRUAURE,N2019AITRE}. %
In \python, the \pythonilIdx{re} module offers functionality work with regular expressions~\cite{K2024REH,PSF2024RREO}.%
}%
}%
%
%
\newglossaryentry{rdb}{%
name={relational database},%
description={%
A relational \pgls{db} is a \pgls{db} that organizes data into rows~(tuples, records) and columns~(attributes), which collectively form tables~(relations) where the data points are related to each other~\cite{I2021WIARDB,C1970ARMODFLSDB,SC1975OTPOARADI,T2018ISARD,H2016RDDAI,HM2024IMARD,databases}.%
}%
}%
%
%
\newglossaryentry{server}{%
text={server},%
name={server},%
sort={server},%
description={%
In a \pgls{clientServerArchitecture}, the \pgls{server} is a process that fulfills the requests of the \pglspl{client}. %
It usually waits for incoming communication carring the requests from the \pglspl{client}. %
For each request, it takes the necessary actions, performs the required computations, and then sends a response with the result of the request. %
Typical examples for \pglspl{server} are web servers~\cite{C2022HAFTLS} in the internet as well as \pglspl{dbms}. %
It is also common to refer to the computer running the \pgls{server} processes as \pgls{server} as well, i.e., to call it the \inQuotes{\pgls{server} computer~\cite{L2022MUS}.}%
}%
}%
%
%
\newglossaryentry{significand}{%
name={significand},%
description={%
The significand is the part of a floating point number that stores the digits of the number (in binary representation). %
In the 64~bit double precision IEEE~Standard 754 floating point number layout~\cite{IEEE2019ISFFPA,H1997IS7FPN}, the exponent is 52~bits.%
\optionalRef{sec:howFloatingPointNumbersWork}{ See \cref{sec:howFloatingPointNumbersWork}.}%
}%
}%
%
%
\newglossaryentry{signature}{%
name={signature},%
description={%
\pythonIdx{function!signature}%
The signature of a function refers to the parameters and their types, the return type, and the exceptions that the function can raise~\cite{M2023SF}. %
In \python, the function~\pythonilIdx{signature} of the module~\pythonilIdx{inspect} provides some information about the signature of a function~\cite{PEP362}.%
}%
}%
%
%
\newglossaryentry{signBit}{%
name={sign bit},%
description={%
The sign bit indicates whether a floating point number is positive or negative in the 64~bit double precision IEEE~Standard 754 floating point number layout~\cite{IEEE2019ISFFPA,H1997IS7FPN}.%
\optionalRef{sec:howFloatingPointNumbersWork}{ See \cref{sec:howFloatingPointNumbersWork}.}%
}%
}%
%
%
\newglossaryentry{stackTrace}{%
name={stack trace},%
description={%
A stack trace gives information the way in which one function invoked another. %
The term comes from the fact that the data needed to implement function calls is stored in a stack data structure~\cite{K1997FA}. %
The data for the most recently invoked function is on top, the data of the function that called is right below, the data of the function that called that one comes next, and so on. %
Printing a stack trace can be very helpful when trying to find out where an \pythonilIdx{Exception} occurred.%
\optionalRef{sec:exceptions}{ See, for instance, \cref{sec:exceptions}.}%
}%
}%
%
%
\newglossaryentry{stderrs}{%
name={standard error stream},%
description={%
The \acrfull{stderr} is one of the three pre-defined streams of a console process (together with the \gls{stdins} and the \gls{stdouts})~\cite{J2024PTOGBSI8IS12ESSSSIS}. %
It is the text stream to which the process writes information about errors and exceptions. %
If an uncaught \pythonilIdx{Exception} is raised in \python\ and the program terminates, then this information is written to \pgls{stderr}. %
If you run a program in a \pgls{terminal}, then the text that a process writes to its \pgls{stderr} appears in the console.%
}%
}%
%
\newglossaryentry{stdins}{%
name={standard input stream},%
description={%
The \acrfull{stdin} is one of the three pre-defined streams of a console process (together with the \pgls{stdouts} and the \pgls{stderrs})~\cite{J2024PTOGBSI8IS12ESSSSIS}. %
It is the text stream from which the process reads its input text, if any. %
The \python\ instruction \pythonilIdx{input} reads from this stream. %
If you run a program in a \pgls{terminal}, then the text that you type into the terminal while the process is running appears in this stream.%
}%
}%
%
\newglossaryentry{stdouts}{%
name={standard output stream},%
description={%
The \acrfull{stdout} is one of the three pre-defined streams of a console process (together with the \pgls{stdins} and the \pgls{stderrs})~\cite{J2024PTOGBSI8IS12ESSSSIS}. %
It is the text stream to which the process writes its normal output. %
The \pythonilIdx{print} instruction of \python\ writes text to this stream. %
If you run a program in a \pgls{terminal}, then the text that a process writes to its \pgls{stdout} appears in the console.%
}%
}%
%
\newglossaryentry{sudo}{%
text={\texttt{sudo}},%
name={sudo},%
sort={sudo},%
description={%
In order to perform administrative tasks such as installing new software under \linux, root (or ``super'') user privileges as needed~\cite{CN2020ULB}. %
A normal user can execute a program in the \pgls{terminal} as super user by pre-pending \bashil{sudo}, often referred to as ``super user do.'' %
This requires the root password.%
}%
}%
%
%
\newglossaryentry{terminal}{%
name={terminal},%
plural={terminals},%
description={%
A terminal is a text-based window where you can enter commands and execute them~\cite{CN2020ULB,B2022ELATCL}. %
Knowing what a terminal is and how to use it is very essential in any programming- or system administration-related task. %
If you want to open a terminal under \windows, you can \windowsTerminal. %
\optionalRef{fig:installingPythonWindows01openTerminal}{This is shown, e.g., in \cref{fig:installingPythonWindows01openTerminal}. }%
Under \ubuntu\ \linux, \ubuntuTerminal\ opens a terminal, which then runs a \bash\ shell inside.%
}%
}%
%
%
\newglossaryentry{typeHint}{%
name={type hint},%
plural={type hints},%
description={%
are annotations that help programmers and static code analysis tools such as \mypy\ to better understand what type a variable or function parameter is supposed to be~\cite{PEP484,PEP482}. %
\python\ is a dynamically typed programming language where you do not need to specify the type of, e.g., a variable. %
This creates problems for code analysis, both automated as well as manual: %
For example, it may not always be clear whether a variable or function parameter should be an integer or floating point number. %
The annotations allow us to explicitly state which type is expected. %
They are \emph{ignored} during the program execution. %
They are a basically a piece of documentation.%
\optionalRef{sec:varTypeHints}{ See \cref{sec:varTypeHints}.}%
}%
}%
%
%
\newglossaryentry{utf8}{%
name={UTF-8},%
description={%
The \acrfull{UTF8} is one standard for encoding \pgls{unicode} characters into a binary format that can be stored in files~\cite{Y2003UATFOI1,ISOIEC106462020ITUCCSU}. %
It is the world wide web's most commonly used character encoding, where each character is represented by one to four bytes. %
It is backwards compatible with ASCII\optionalRef{sec:unicodeChars}{~(see~\cref{sec:unicodeChars})}.%
}%
}%
%
%
\newglossaryentry{unicode}{%
name={Unicode},%
description={%
A standard for assigning characters to numbers~\cite{TUC2023U1510,TUC2023U151ACS,ISOIEC106462020ITUCCSU}. %
The Unicode standard supports basically all characters from all languages that are currently in use, as well as many special symbols. %
It is the predominantly used way to represent characters in computers and is regularly updated and improved.%
}%
}%
%
%
\newglossaryentry{vcs}{%
name={Version Control System},%
plural={Version Control Systems},%
description={%
A \acrfull{VCS} is a software which allows you to manage and preserve the historical development of your program code~\cite{T2024BGAGVCPMATFTND}. %
A distributed \pgls{VCS} allows multiple users to work on the same code and upload their changes to the server, which then preserves the change history.
The most popular distributed \pgls{VCS} is \pgls{git}.%
}%
}%
%
%
\newglossaryentry{virtualEnvironment}{%
name={virtual environment},%
description={%
A virtual environment is a directory that contains a local \python\ installation~\cite{PSF2024VEAP,PEP405}. %
It comes with its own package installation directory. %
Multiple different virtual environments can be installed on a system. %
This allows different applications to use different versions of the same packages without conflict, because we can simply install these applications into different \pglspl{virtualEnvironment}.%
}%
}%
%
%
\newglossaryentry{xAxis}{%
name={\ensuremath{x}\nobreakdashes-axis},%
plural={\ensuremath{x}\nobreakdashes-axes},%
sort={x-axis},
description={%
The \ensuremath{x}\nobreakdashes-axis is the horizontal axis of a two-dimensional coordinate system, often referred to abscissa.%
}%
}%
%
%
\newglossaryentry{xml}{%
name={XML},%
plural={XML},%
description={%
\sloppy%
The \acrfull{XML} is a text-based language for storing and transporting of data~\cite{BPSMM2008EMLX1FE,K2019ITXJY,CH2013XFCAMLTMC}. %
It allows you to define elements in the form \xmlil{<myElement myAttr="x">...text..</myElement>}. %
Different from \pgls{csv}, elements in XML can be hierarchically nested, like~\textil{<a><b><c>test</c></b><b>bla</b></a>}, and thus easily represent tree structures. %
XML is one of most-used data interchange formats. %
To process XML in \python, use the \pythonilIdx{defusedxml} library~\cite{H2021D0XBPFPSM}, as it protects against several security issues.%
}%
}%
%
\newglossaryentry{yaml}{
name={YAML},%
plural={YAML},%
description={%
\acrfull{YAML} is a human-friendly data serialization language for all programming languages~\cite{DNMAASBE2021YAMLYV1,K2019ITXJY,CGTYB2022YFFDCAIE}. %
It is widely used for configuration files in the DevOps environment. %
See \url{https://yaml.org} for more information.%
}%
}%
%

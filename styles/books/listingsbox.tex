%%
%% Listings and Listing Styles.
%%
%%
%% Use texgit to place a listing.
%% #1 the git repository
%% #2 the local path
%% #3 the post-processing
%% #4 the label (lst: will be pre-pended)
%% #5 the caption
%% #6 the style to use
\protected\gdef\gitCode#1#2#3#4#5#6{%
\gitLoad{lst:#4}{#1}{#2}{#3}%
\begin{figure}[tb]%
\lstinputlisting[label={lst:#4}#6,caption={#5~(\href{\gitUrl{lst:#4}}{src})}]{\gitFile{lst:#4}}%
\end{figure}%
}%
%% Use texgit to place a listing with program output.
%% #1 the git repository
%% #2 the local path
%% #3 the command
%% #4 the label (exec: will be pre-pended)
%% #5 the additional parameters, if any
%% #6 the style to use
\protected\gdef\gitOutputWithStyle#1#2#3#4#5#6{%
\gitExec{exec:#4}{#1}{#2}{#3}%
\lstinputlisting[style=#6#5]{\gitFile{exec:#4}}%
}%
%
%
%
%
%%
%% Use texgit to place a SQL listing.
%% #1 the git repository
%% #2 the local path
%% #3 the label (lst: will be pre-pended)
%% #4 the caption
\protected\gdef\gitSQL#1#2#3#4{%
\gitCode{#1}{#2}{}{#3}{#4}{,style=sql_style}%
}%
%
%%% Set the SQL output format, because sometimes we want to have different formats in the output
\protected\gdef\gitSQLAndOutputFormat#1{\xdef\@gitSQLAndOutputFormat{#1}}%
\protected\gdef\gitSQLAndOutputFormatDefault{\gitSQLAndOutputFormat{tool_style_sql}}%
\gitSQLAndOutputFormatDefault%
%
%% Use texgit to place an SQL listing code and the program output.
%% #1 the git repository
%% #2 the path to the directory
%% #3 the path to the file inside the directory
%% #4 the database to invoke it on, if any
%% #5 the user to use
%% #6 the password, if any
%% #7 the script to invoke
%% #8 the label (lst: will be prepended to the program, exec: will be
%%    pre-pended to the output)
%% #9 the caption
\protected\gdef\gitSQLAndOutput#1#2#3#4#5#6#7#8#9{%
\expandafter\expandafter\expandafter\edef\expandafter\csname @gitSQLAndOutputFormat:exec:#8\endcsname{\@gitSQLAndOutputFormat}%
\gitSQLAndOutputFormatDefault%
%
\gitLoad{lst:#8}{#1}{#2/#3}{}%
\gitExec{exec:#8}{#1}{.}{_scripts_/#7 #2 #3 #4 #5 #6}%
%
\begin{figure}[tb]%
\centering%
\lstinputlisting[label={lst:#8},style=sql_style,caption={#9~(stored in file~\href{\gitUrl{lst:#8}}{\textil{#3}}; output in~\cref{exec:#8})}]{\gitFile{lst:#8}}%
%
\medskip%
%
\edef\@@@@gitOutputFmt{\csname @gitSQLAndOutputFormat:exec:#8\endcsname}%
\lstinputlisting[label={exec:#8},style=\@@@@gitOutputFmt,caption={The \gls{stdout} resulting from the \sql\ statements in~\href{\gitUrl{lst:#8}}{\textil{#3}} given in~\cref{lst:#8}.}]{\gitFile{exec:#8}}%
\end{figure}%
}%
%
%
%%
%% The python syntax environment.
%% Warning: If we do not use the minipage, this may or may not
%% cause strange errors like:
%% "! Argument of ? has an extra }."
\lstnewenvironment{pythonSyntax}[1][true]{%
\center%
\minipage{\linewidth}%
\let\@old@lst@visiblespace\lst@visiblespace%
\def\lst@visiblespace{{\color{listing-spaces}\@old@lst@visiblespace}}%
\lstset{%
style=python_style,%
showspaces=#1%
}}{%
\let\lst@visiblespace\@old@lst@visiblespace%
\endminipage%
\endcenter%
}%
%

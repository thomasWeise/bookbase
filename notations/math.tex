%%% Mathematical Notation
%%%
\protected\gdef\decSep{\hspace*{0.175em}}%
%
\protected\gdef\mathSpace#1{\ensuremath{\mathbb{#1}}}%
%
\newSymbol{naturalNumbersZ}{\ensuremath{\mathSpace{N}_0}}{N0}{%
the set of the natural numbers \emph{including}~0, i.e., 0, 1, 2, 3, and so on. %
It holds that~$\naturalNumbersZ\subset\integerNumbers$%
}%
\newSymbol{naturalNumbersO}{\ensuremath{\mathSpace{N}_1}}{N1}{%
the set of the natural numbers \emph{excluding}~0, i.e., 1, 2, 3, 4, and so on. %
It holds that~$\naturalNumbersO\subset\integerNumbers$%
}%
\newSymbol{integerNumbers}{\ensuremath{\mathSpace{Z}}}{Z}{%
the set of the integers numbers including positive and negative numbers and~0, i.e., {\dots}, -3, -2, -1, 0, 1, 2, 3, {\dots}, and so on. %
It holds that~$\integerNumbers\subset\realNumbers$%
}%
\newSymbol{realNumbers}{\mathSpace{R}}{R}{the set of the real numbers}%
\newSymbol{realNumbersP}{\ensuremath{\mathSpace{R}^+}}{R+}{%
the set of the positive real numbers, i.e., $\mathSpace{R}^+=\{x\in\realNumbers:x>0\}$}%
%
\newSymbol{rationalNumbers}{\mathSpace{Q}}{Q}{%
the set of the rational numbers, i.e., the set of all numbers that can be the result of~$\frac{a}{b}$ with~$a,b\in\integerNumbers$ and~$b\neq0$. %
$a$~is called the \pgls{numerator} and $b$~is called the \pgls{denominator}. %
It holds that~$\integerNumbers\subset\rationalNumbers$ and $\rationalNumbers\subset\realNumbers$%
}%
%
\protected\gdef\intRange#1#2{\ensuremath{#1\gls{intSetDots}#2}}%
\newglossaryentry{intSetDots}{%
type={symbols},
name={$i..j$},%
text={..},%
sort={..},%
description={\sloppy%
with $i,j\in\integerNumbers$ and $i\leq j$ is the set that contains all integer numbers in the inclusive range from~$i$ to~$j$. %
For example, \mbox{$5..9$}~is equivalent to~\mbox{$\{5, 6, 7, 8, 9\}$}}%
}%
%
%
\newFunc{bigO}{\ensuremath{\mathcal{O}}}{O(g(x))}{g(x)}{%
\mbox{If $f(x)={\mathcal{O}}(g(x))$,} then there exist positive numbers~$x_0\in\realNumbersP$ and~$c\in\realNumbersP$ such that~\mbox{$f(x)\leq c*g(x)\forall x\geq x_0$}~\cite{B1894DAZDVPB,L1909HDLVDVDP}. %
In other words, ${\mathcal{O}}(g(x))$~describes an upper bound for function growth}%
%
\newSymbol{numberPi}{\ensuremath{\pi}}{$\pi$}{%
is the ratio of the circumference~$U$ of a circle and its diameter~$d$, i.e., $\pi=U/d$. %
$\pi\in\realNumbers$ is an irrational and transcendental number~\cite{N1939TTOP,APM1991TOEAP,F2011TTOEAP}, which is approximately~$\pi\approx3.141\decSep592\decSep653\decSep589\decSep793\decSep238\decSep462\decSep643$. %
In \python, it is provided by the \pythonilIdx{math} module as constant \pythonilIdx{pi} with value~\pythonilIdx{3.141592653589793}}%
%
\newSymbol{numberE}{\ensuremath{e}}{e}{%
is Euler's number~\cite{E1737DFCD}, the base of the natural logarithm. %
$e\in\realNumbers$ is an irrational and transcendental number~\cite{APM1991TOEAP,F2011TTOEAP}, which is approximately~$e\approx2.718\decSep281\decSep828\decSep459\decSep045\decSep235\decSep360$. %
In \python, it is provided by the \pythonilIdx{math} module as constant \pythonilIdx{e} with value~\pythonilIdx{2.718281828459045}}%
%
\newSymbol{numberGoldenRatio}{\ensuremath{\phi}}{$\phi$}{%
The golden ratio (or golden section) is the irrational number~$\frac{1+\sqrt{5}}{2}$. %
It is the ratio of a line segment cut into two pieces of different lengths such that the ratio of the whole segment to that of the longer segment is equal to the ratio of the longer segment to the shorter segment~\cite{CEOEB2024GR,EHF2008EEOGTGOJLH11FEEEELIEILHIATBG11EAPWMETBFR}. %
The golden ratio is approximately~$\phi\approx1.618\decSep033\decSep988\decSep749\decSep894\decSep848\decSep204\decSep586\decSep834$~\cite{S2024DEOGRPOT}. %
Represented as \pythonil{float} in \python, its value is~\pythonil{1.618033988749895}}%
%
\protected\gdef\factorial#1{\ensuremath{#1\gls{factorialMark}}}%
\newglossaryentry{factorialMark}{%
type={symbols},
name={$i!$},%
text={!},%
sort={!},%
description={\sloppy%
The factorial~$a!$ of a natural number~$a\in\naturalNumbersO$ is the product of all positive natural numbers less than or equal to~$a$, i.e., $a!=1*2*3*4*\dots*(a-1)*a$~\cite{D1991TEHOTFF,CB2022FBDOTFF,L2015ANKOFF}. %
See also \cref{eq:factorial} and \cref{lst:functions:def_factorial}%
}}%
%
%
\newFunc{attrDomain}{\ensuremath{\operatorname{dom}}}{dom(a)}{a}{%
the domain of an attribute~$a$\optionalRef{def:attributeDomain}{, see \cref{def:attributeDomain}}}%
%

%%
%% Algorithm Commands
%%
%
\RequirePackage[ruled,vlined,noend]{algorithm2e}%
\SetAlgorithmName{Algorithm}{Algorithm}{List of Algorithms}%
\SetKwComment{comment}{$\triangleright$\ }{}%
\renewcommand{\CommentSty}[1]{{\itshape\footnotesize\color{algorithm-comment}#1}}%
\SetAlgoLined%
\let\@oldKwStyle\KwSty%
\renewcommand{\KwSty}[1]{{\color{algorithm-keywords}\@oldKwStyle{#1}}}%
%
%
%% A separator between two instructions to appear on the same line
\xdef\aSep{;~~~}%
%%
%% The variable style.
\protected\gdef\aVar#1{\ensuremath{\mathit{#1}}}%
%
%% Assign value #2 to variable #1.
\protected\gdef\aAssign#1#2{\ensuremath{#1\gets#2}}%
%
%% Element #2 of array #1
\protected\gdef\aArrayIndex#1#2{\mbox{\ensuremath{#1\!\left[#2\right]}}}%
%
\ifIsBook\expandafter\@firstoftwo\else\expandafter\@secondoftwo\fi{%
%%
%% Algorithm Commands
%%
%
\RequirePackage[ruled,vlined,noend]{algorithm2e}%
\SetAlgorithmName{Algorithm}{Algorithm}{List of Algorithms}%
\SetKwComment{comment}{$\triangleright$\ }{}%
\renewcommand{\CommentSty}[1]{{\itshape\footnotesize\color{algorithm-comment}#1}}%
\SetAlgoLined%
\let\@oldKwStyle\KwSty%
\renewcommand{\KwSty}[1]{{\color{algorithm-keywords}\@oldKwStyle{#1}}}%
%
%
%% A separator between two instructions to appear on the same line
\xdef\aSep{;~~~}%
%%
%% The variable style.
\protected\gdef\aVar#1{\ensuremath{\mathit{#1}}}%
%
%% Assign value #2 to variable #1.
\protected\gdef\aAssign#1#2{\ensuremath{#1\gets#2}}%
%
%% Element #2 of array #1
\protected\gdef\aArrayIndex#1#2{\mbox{\ensuremath{#1\!\left[#2\right]}}}%
%
\ifIsBook\expandafter\@firstoftwo\else\expandafter\@secondoftwo\fi{%
%%
%% Algorithm Commands
%%
%
\RequirePackage[ruled,vlined,noend]{algorithm2e}%
\SetAlgorithmName{Algorithm}{Algorithm}{List of Algorithms}%
\SetKwComment{comment}{$\triangleright$\ }{}%
\renewcommand{\CommentSty}[1]{{\itshape\footnotesize\color{algorithm-comment}#1}}%
\SetAlgoLined%
\let\@oldKwStyle\KwSty%
\renewcommand{\KwSty}[1]{{\color{algorithm-keywords}\@oldKwStyle{#1}}}%
%
%
%% A separator between two instructions to appear on the same line
\xdef\aSep{;~~~}%
%%
%% The variable style.
\protected\gdef\aVar#1{\ensuremath{\mathit{#1}}}%
%
%% Assign value #2 to variable #1.
\protected\gdef\aAssign#1#2{\ensuremath{#1\gets#2}}%
%
%% Element #2 of array #1
\protected\gdef\aArrayIndex#1#2{\mbox{\ensuremath{#1\!\left[#2\right]}}}%
%
\ifIsBook\expandafter\@firstoftwo\else\expandafter\@secondoftwo\fi{%
%%
%% Algorithm Commands
%%
%
\RequirePackage[ruled,vlined,noend]{algorithm2e}%
\SetAlgorithmName{Algorithm}{Algorithm}{List of Algorithms}%
\SetKwComment{comment}{$\triangleright$\ }{}%
\renewcommand{\CommentSty}[1]{{\itshape\footnotesize\color{algorithm-comment}#1}}%
\SetAlgoLined%
\let\@oldKwStyle\KwSty%
\renewcommand{\KwSty}[1]{{\color{algorithm-keywords}\@oldKwStyle{#1}}}%
%
%
%% A separator between two instructions to appear on the same line
\xdef\aSep{;~~~}%
%%
%% The variable style.
\protected\gdef\aVar#1{\ensuremath{\mathit{#1}}}%
%
%% Assign value #2 to variable #1.
\protected\gdef\aAssign#1#2{\ensuremath{#1\gets#2}}%
%
%% Element #2 of array #1
\protected\gdef\aArrayIndex#1#2{\mbox{\ensuremath{#1\!\left[#2\right]}}}%
%
\ifIsBook\expandafter\@firstoftwo\else\expandafter\@secondoftwo\fi{%
\input{\bookbaseDir/styles/books/algorithms.tex}%
}{}%
%
%
}{}%
%
%
}{}%
%
%
}{}%
%
\ifIsSlides\expandafter\@firstoftwo\else\expandafter\@secondoftwo\fi{%
%%
%% Algorithm Commands
%%
%
\RequirePackage[ruled,vlined,noend]{algorithm2e}%
\SetAlgorithmName{Algorithm}{Algorithm}{List of Algorithms}%
\SetKwComment{comment}{$\triangleright$\ }{}%
\renewcommand{\CommentSty}[1]{{\itshape\footnotesize\color{algorithm-comment}#1}}%
\SetAlgoLined%
\let\@oldKwStyle\KwSty%
\renewcommand{\KwSty}[1]{{\color{algorithm-keywords}\@oldKwStyle{#1}}}%
%
%
%% A separator between two instructions to appear on the same line
\xdef\aSep{;~~~}%
%%
%% The variable style.
\protected\gdef\aVar#1{\ensuremath{\mathit{#1}}}%
%
%% Assign value #2 to variable #1.
\protected\gdef\aAssign#1#2{\ensuremath{#1\gets#2}}%
%
%% Element #2 of array #1
\protected\gdef\aArrayIndex#1#2{\mbox{\ensuremath{#1\!\left[#2\right]}}}%
%
\ifIsBook\expandafter\@firstoftwo\else\expandafter\@secondoftwo\fi{%
%%
%% Algorithm Commands
%%
%
\RequirePackage[ruled,vlined,noend]{algorithm2e}%
\SetAlgorithmName{Algorithm}{Algorithm}{List of Algorithms}%
\SetKwComment{comment}{$\triangleright$\ }{}%
\renewcommand{\CommentSty}[1]{{\itshape\footnotesize\color{algorithm-comment}#1}}%
\SetAlgoLined%
\let\@oldKwStyle\KwSty%
\renewcommand{\KwSty}[1]{{\color{algorithm-keywords}\@oldKwStyle{#1}}}%
%
%
%% A separator between two instructions to appear on the same line
\xdef\aSep{;~~~}%
%%
%% The variable style.
\protected\gdef\aVar#1{\ensuremath{\mathit{#1}}}%
%
%% Assign value #2 to variable #1.
\protected\gdef\aAssign#1#2{\ensuremath{#1\gets#2}}%
%
%% Element #2 of array #1
\protected\gdef\aArrayIndex#1#2{\mbox{\ensuremath{#1\!\left[#2\right]}}}%
%
\ifIsBook\expandafter\@firstoftwo\else\expandafter\@secondoftwo\fi{%
%%
%% Algorithm Commands
%%
%
\RequirePackage[ruled,vlined,noend]{algorithm2e}%
\SetAlgorithmName{Algorithm}{Algorithm}{List of Algorithms}%
\SetKwComment{comment}{$\triangleright$\ }{}%
\renewcommand{\CommentSty}[1]{{\itshape\footnotesize\color{algorithm-comment}#1}}%
\SetAlgoLined%
\let\@oldKwStyle\KwSty%
\renewcommand{\KwSty}[1]{{\color{algorithm-keywords}\@oldKwStyle{#1}}}%
%
%
%% A separator between two instructions to appear on the same line
\xdef\aSep{;~~~}%
%%
%% The variable style.
\protected\gdef\aVar#1{\ensuremath{\mathit{#1}}}%
%
%% Assign value #2 to variable #1.
\protected\gdef\aAssign#1#2{\ensuremath{#1\gets#2}}%
%
%% Element #2 of array #1
\protected\gdef\aArrayIndex#1#2{\mbox{\ensuremath{#1\!\left[#2\right]}}}%
%
\ifIsBook\expandafter\@firstoftwo\else\expandafter\@secondoftwo\fi{%
\input{\bookbaseDir/styles/books/algorithms.tex}%
}{}%
%
%
}{}%
%
%
}{}%
%
%
}{}%
%
%%
%% We have two commands for comments in algorithms.
%%
%% The command `'\algoCmtInline` offers a right-justified comment to be used
%% inside primitives, such as \If or \For loops.
%% It takes a single parameter, namely the comment text.
%% `\algoCmtEndline` is to be used at the end of normal lines (and ends them).
%%
%% Commands can be enabled using \algoCommentsOn, which is the default.
%% They can also be disabled using \algoCommentsOff.
%
%% Turn on all comments: The comments issued via `\algoCmt`
%% will be printed.
\protected\gdef\algoCommentsOn{%
\protected\gdef\algoCmtInline##1{\comment*[f]{##1}}%
\protected\gdef\algoCmtEndline##1{\comment*[r]{##1}}%
}%
\algoCommentsOn%
%
%% Turn on all comments: The comments issued via `\algoCmt`
%% will NOT be printed.
\protected\gdef\algoCommentsOff{%
\protected\gdef\algoCmtInline##1{}%
\protected\gdef\algoCmtEndline##1{\;}%
}%
